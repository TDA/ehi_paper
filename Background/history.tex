\section{History of E-Mail Injection}

E-Mail Header Injection seems to have been first documented over a decade ago, in a late 2004 article on phpsecure.info \cite{Tobozo} accredited to user \lstinline|tobozo@phpsecure.info| describing how this vulnerability existed in the reference implementation of the \dq{\texttt{mail}} function in PHP, and how it can be exploited. More recently, a blog post by Damon Kohler \cite{DK} and an accompanying wiki article \cite{Injection} describe the attack vector and outline a few defense measures for the same.

As this vulnerability was initially found in the \emph{mail()} function of PHP, E-Mail Header Injection can be traced to as early as the beginning of the 2000's, present in the \emph{mail()} implementation of PHP 4.0. 

The vulnerability was also described briefly (less than a page) by Stuttard and Pinto in their widely acclaimed book, ``\emph{The Web Application Hacker's Handbook: Discovering and Exploiting Security Flaws}'' \cite{stuttard2011web}. 
A concise timeline of the vulnerability is presented in Table~\ref{tab:history}.

An example of the vulnerable code written in PHP is shown in Listing~\ref{code:phpemi}. This code takes in user input from the PHP superglobal \dq{\texttt{\$\_REQUEST[\textquotesingle email\textquotesingle]}}, and stores it in the variable \dq{\texttt{\$from}}, which is later passed to the \dq{\texttt{mail}} function to construct and send the e-mail.
\begin{lstlisting}[language=PHP,caption={PHP program with e-mail
      header injection vulnerability.},label={code:phpemi}, float]
$from = $_REQUEST['email'];
$subject = 'Hello XYZ';
$message = 'We need you to reset your password';
$to = 'xyz@example.com';
// example attack string injected as for
// $_REQUEST['email'] => 
// 'abc@example.com\nCC:spc@example.com'
$retValue = mail($to, $subject, $message, "From: $from");
// E-Mail gets sent to both 
// xyz@example.com AND spc@example.com
\end{lstlisting}

	
When this code is given the malicious input \dq{\texttt{\lstinline{sai@sai.com\\nBCC:spc@spc.com}}} as the value of the \dq{\texttt{\$\_REQUEST[\textquotesingle email\textquotesingle]}}, it generates the SMTP Headers shown in Listing~\ref{code:smtpheaders}. It can be seen that the `CC' (carbon copy) header that we injected appears as part of the resulting SMTP message. This will make the e-mail get sent to the e-mail address specified as part of the `CC' as well.
\lstset{language=HTML,caption={SMTP headers generated by a PHP mailing script.},label={code:smtpheaders}}
\begin{lstlisting}
Received: from mail.ourdomain.com ([62.121.130.29])
	by xyz.com (Postfix) with ESMTP id 5A08E52C0154
	for <s@s.com>; Sun, 20 Mar 2016 13:56:58 -0700 (MST)
To: s@s.com
Subject: Hello XYZ
CC: spc@spc.com
Date: Sun, 20 Mar 2016 13:56:58 -0700 (MST)

We need you to reset your password
\end{lstlisting}
\begin{table}[tbp]
  \scriptsize
	\centering
	\scalebox{0.85}{
	\begin{tabular}{|c|p{5.5cm}|c|}
		\hline
		\multicolumn{1}{|c|}{\textbf{CVE No.}} & 
		\multicolumn{1}{c|}{\textbf{Affected Software}} &
		\multicolumn{1}{c|}{\textbf{Year}}\\
		\hline
		{2002-1575} & {cgiemail} & {2004}\\
		\hline
		{2002-1771} & {FormMail 1.9} & {2005}\\
		\hline
		{2002-1917} & {Geeklog 1.35} & {2005}\\
		\hline
		{2005-0493} & {Biz Mail Form <=2.2} & {2005}\\
		\hline
		{2005-2854} & {thesitewizard.com} & {2005}\\
		\hline
		{2005-3883} & {PHP mb\_send\_mail} & {2005}\\
		\hline
		{2006-0631} & {Perl mailback.pl} & {2006}\\
		\hline
		{2006-0712} & {Squishdot 1.5.0} & {2006}\\
		\hline
		{2006-1225} & {Drupal 4.5.0-4.5.8 and 4.6.0-4.6.8} & {2006}\\
		\hline
		{2006-1305} & {Microsoft Outlook 2000, 2002-03} & {2006}\\
		\hline
		{2006-2159} & {Russcom Network} & {2006}\\
		\hline
		{2006-2943} & {CGI-RESCUE WebFORM 4.1} & {2006}\\
		\hline
		{2006-2944} & {CGI-RESCUE FORM2MAIL 1.21} & {2006}\\
		\hline
		{2006-3171} & {CS-Forum <=0.82} & {2006}\\
		\hline
		{2006-3473} & {Drupal Module <=1.8.2.2} & {2006}\\
		\hline
		{2006-4344} & {CGI-Rescue Mail} & {2006}\\
		\hline
		{2006-7020} & {phpwcms 1.2.5-DEV} & {2007}\\
		\hline
		{2006-7087} & {Dotdeb PHP} & {2007}\\
		\hline
		{2007-1718} & {PHP 4.0-4.4.6 and 5.0-5.2.1} & {2007}\\
		\hline
		{2007-1898} & {Jetbox CMS 2.1} & {2007}\\
		\hline
		{2007-1900} & {FILTER\_VALIDATE\_EMAIL PHP} & {2007}\\
		\hline
		{2007-2731} & {Jetbox CMS 2.1} & {2007}\\
		\hline
		{2008-2105} & {Bugzilla} & {2008}\\
		\hline
		{2009-1469} & {IceWarp} & {2009}\\
		\hline
		{2008-7281} & {OTRS - Open Ticket Request System} & {2011}\\
		\hline
		{2014-2957} & {Exim} & {2014}\\
		\hline
		{2015-8476} & {PHPMailer} & {2015}\\
		\hline
		{2016-4803} & {dotCMS} & {2016}\\
		\hline

	\end{tabular}
	}
	\caption{History of software found in Common Vulnerabilities and
      Exposures database affected by e-mail header injection
      vulnerability}
    \vspace{-5ex}
	\label{tab:history}
\end{table}
%TODO Adam: added this table

