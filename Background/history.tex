\subsection{History of \ehi}

We found the first \ehi description in a late 2004 article on phpsecure.info~\cite{Tobozo} accredited to user \lstinline|tobozo| describing how an \ehi vulnerability existed in the implementation of the \texttt{mail()} function in PHP and how it can be exploited. More recently, a blog post by Damon Kohler~\cite{DK} and an accompanying wiki article~\cite{Injection} describe the attack vector and outline few defense measures for \ehi vulnerabilities.

%% As this vulnerability was initially found in the \texttt{mail()} function of PHP, \ehi can be traced to as early as the beginning of the 2000's, present in the \texttt{mail()} implementation of PHP 4.0. 

An example of the vulnerable code written in PHP is shown in Listing~\ref{code:phpemi}. This code takes in user input from the PHP superglobal \texttt{\$\_REQUEST[\textquotesingle email\textquotesingle]}, and stores it in the variable \texttt{\$from}, which is later passed to the \texttt{mail()} function to construct and send the e-mail.

\begin{lstlisting}[language=PHP,caption={PHP program with e-mail
      header injection vulnerability.},label={code:phpemi}, float]
$from = $_REQUEST['email'];
$subject = 'Hello XYZ';
$message = 'We need you to reset your password';
$to = 'xyz@example.com';
$retValue = mail($to, $subject, $message, "From: $from");
\end{lstlisting}



\begin{sloppypar}
When this code is given the malicious input \texttt{\lstinline{abc@example.com\\nCC:spc@example.com}} as the value of the \texttt{\$\_REQUEST[\textquotesingle email\textquotesingle]}, it generates the equivalent SMTP Headers shown in Listing~\ref{code:smtpheaders}. It can be seen that the \texttt{CC} (carbon copy) header that we injected appears as part of the resulting SMTP message. This will make the e-mail get sent to the e-mail address specified as part of the \texttt{CC} as well. 
\begin{lstlisting}[language=HTML,caption={SMTP headers generated by a PHP mailing
  script.},label={code:smtpheaders}, float]
Received: from mail.ourdomain.com ([62.121.130.29])
  by xyz.com (Postfix) with ESMTP id 5A08E52C0154
  for <abc@example.com>; Sun, 20 Mar 2016 13:56:58 -0700 (MST)
From: abc@example.com
To: xyz@example.com
Subject: Hello XYZ
CC: spc@example.com
Date: Sun, 20 Mar 2016 13:56:58 -0700 (MST)

We need you to reset your password
\end{lstlisting}

%\begin{table}[!htbp]
	\centering
	\begin{tabular}{|p{2cm}|p{12cm}|}
		\hline
		\multicolumn{1}{|c|}{\textbf{Year}} & \multicolumn{1}{c|}{\textbf{ Notes}}\\
		\hline

		{Early 2000's } & { PHP 4.0 is released, along with support for the mail() function, which has no protection against E-Mail Header Injection.}\\
		\hline

		{Jul 2004} & { Next Major version of PHP - Version 5.0 releases}\\
		\hline

		{Dec 2004} & { First known article about the vulnerability surfaces on phpsecure.info}\\
		\hline

		{Oct 2007} & {The vulnerability makes its way into a text by Stuttard and Pinto. }\\
		\hline

		{Dec 2008} & {Blog post and accompanying wiki about the header injection attack in detail with examples.}\\
		\hline

		{Apr 2009} & {Bug filed about email.header package to fix the issue on Python Bug Tracker}\\
		\hline

		{Jan 2011} & {Bug fix for Python 3.1, Python 3.2, Python 2.7 for email.header package, backport to older versions not available.}\\
		\hline

		{Sep 2011} & {The vulnerability is described with an example in the 2nd edition of the text by Stuttard and Pinto.}\\
		\hline

		{Aug 2013} & {Acunetix adds E-Mail Header Injection to the list of vulnerabilties they detect, as part of their Enterprise Web Vulnerability Scanner Software.}\\
		\hline

		{May 2014} & {Security Advisory for JavaMail SMTP Header Injection via method setSubject is written by Alexandre Herzog.}\\
		\hline

		{Dec 2015}  & {PHP 7 releases, mail function still unpatched.}\\
		\hline
	\end{tabular}
	\caption[\titlecap{A brief history of e-mail header injection}]{A brief history of e-mail header injection.}
	\label{tab:history}
\end{table}

\end{sloppypar}

%TODO Adam: added this para abt E-Mail Header Injection history from CVE
We gathered data from the Common Vulnerabilities and Exposures (CVE)~\cite{cve} database to get an idea of the distribution of \ehi vulnerabilities reported over time. We found 28 reports of \ehi from the CVE (shown in Table~\ref*{tab:history}).

From the table, it can be seen that even though a lot of \ehi vulnerabilities were found in the earlier years (2005-07), there have been several recently discovered \ehi vulnerabilities which suggests that it is still a very real threat and is relevant to modern web security.

\begin{table}[!htbp]
	\centering
	\begin{tabular}{|p{2cm}|p{12cm}|}
		\hline
		\multicolumn{1}{|c|}{\textbf{Year}} & \multicolumn{1}{c|}{\textbf{ Notes}}\\
		\hline

		{Early 2000's } & { PHP 4.0 is released, along with support for the mail() function, which has no protection against E-Mail Header Injection.}\\
		\hline

		{Jul 2004} & { Next Major version of PHP - Version 5.0 releases}\\
		\hline

		{Dec 2004} & { First known article about the vulnerability surfaces on phpsecure.info}\\
		\hline

		{Oct 2007} & {The vulnerability makes its way into a text by Stuttard and Pinto. }\\
		\hline

		{Dec 2008} & {Blog post and accompanying wiki about the header injection attack in detail with examples.}\\
		\hline

		{Apr 2009} & {Bug filed about email.header package to fix the issue on Python Bug Tracker}\\
		\hline

		{Jan 2011} & {Bug fix for Python 3.1, Python 3.2, Python 2.7 for email.header package, backport to older versions not available.}\\
		\hline

		{Sep 2011} & {The vulnerability is described with an example in the 2nd edition of the text by Stuttard and Pinto.}\\
		\hline

		{Aug 2013} & {Acunetix adds E-Mail Header Injection to the list of vulnerabilties they detect, as part of their Enterprise Web Vulnerability Scanner Software.}\\
		\hline

		{May 2014} & {Security Advisory for JavaMail SMTP Header Injection via method setSubject is written by Alexandre Herzog.}\\
		\hline

		{Dec 2015}  & {PHP 7 releases, mail function still unpatched.}\\
		\hline
	\end{tabular}
	\caption[\titlecap{A brief history of e-mail header injection}]{A brief history of e-mail header injection.}
	\label{tab:history}
\end{table}

