\ehi belongs to the class of command injection vulnerabilities. However, unlike its more popular siblings, SQL injection~\cite{sql1, sql0, sql2}, Cross-Site Scripting~\cite{Injection1, KleinAmit}, or HTTP Header Injection~\cite{sessionride}, relatively little research is available on \ehi vulnerabilities.

As with other command injection vulnerabilities, \ehi is caused by
improper or nonexistent sanitization of user input. If the program
constructs \emails from user input and fails to check for the presence
of \email headers, a malicious user can control the headers of this particular e-mail. \ehi vulnerabilities can be leveraged to enable malicious attacks, including, but not limited to, spoofing or phishing.
