\subsection{Languages Affected}
\label{languages}
This section describes the popular programming languages that contain \ehi vulnerabilities in their standard email libraries. This section is not intended as a complete reference of vulnerable functions and methods, but rather as a guide that specifies which parts of the language are vulnerable.

\noindent{\textbf{PHP}} was one of the first languages found to be vulnerable to \ehi in its implementation of the \texttt{mail()} function at the time of release of PHP~4.0. The early finding of this vulnerability can be attributed in part to the success and popularity of the language for creating web pages. According to w3techs~\cite{W3techs}, PHP is used by 81.9\% of all the websites.

After 13 further iterations of PHP since the 4.0 release (the current version is 7.1), the \texttt{mail()} function is yet to be fixed after 15 years. However, it is specified in the PHP documentation~\cite{PHPDocs} that the \texttt{mail()} function does not protect against \ehi.
A working code sample of the vulnerability, written in PHP~5.6 (latest well-supported version), is shown in  Listing~\ref{code:phpemi}.

\noindent{\textbf{Python}}
A bug was filed about an \ehi vulnerability in Python's implementation of the \texttt{email.header} library and the header parsing functions allowing newlines in early 2009, which was followed by a partial patch in early 2011.

Unfortunately, the bug fix was only for the \texttt{email.header} package, and thus exists in other frequently used packages such as \texttt{email.parser}, where both the classic \texttt{Parser()} and the newer \texttt{FeedParser()} contain \ehi vulnerabilities even in the latest versions: \texttt{2.7.11} and \texttt{3.5}. The bug fix was also not backported to older versions of Python.
There is no mention of the vulnerability in the Python documentation for either library.
A working code sample of the vulnerability, written in Python 2.7.11, is shown in Listing~\ref{code:pyemi}.

\begin{lstlisting}[language=Python,caption={Python program with e-mail
      header injection vulnerability.},label={code:pyemi}, float]
from email.parser import Parser
import cgi
form = cgi.FieldStorage()
to = form["email"] # input() exhibits 
# the same behavior
msg = """To: """ + to + """\n
From: <user@example.com>\n
Subject: Test message\n\n
Body would go here\n"""

f = FeedParser() # Parser.parsestr() 
# also contains the same vulnerability
f.feed(msg)
headers = FeedParser.close(f)

# attack string => 
# 'abc@example.com\nBCC:spc@example.com'
# for form["email"]

# to:abc@example.com AND bcc:spc@example.com 
# are added to the headers
print 'To: %s' % headers['to']
print 'BCC: %s' % headers['bcc']
\end{lstlisting}


\noindent{\textbf{Java}} has a bug report about \ehi filed against its \texttt{JavaMail} API. A detailed write-up by Alexandre Herzog~\cite{Herzog.2014} contains a proof-of-concept program that exploits the API to inject headers.

\noindent{\textbf{Ruby}} 
From our preliminary testing Ruby's built-in \texttt{Net::SMTP} library also has an \ehi library. This is not documented on the library's homepage.
%A working code sample of the vulnerability, written in Ruby 2.0.0 (the latest stable version at the time of writing), is shown in Listing~\ref{code:rubyemi}.

%\begin{lstlisting}[language=Ruby,caption={Ruby program with e-mail
      header injection vulnerability.},label={code:rubyemi}, float]
require 'sinatra'
require 'net/smtp'

get '/hello' do
email = params[:email]

message = """
From: Sai <schand31@asu.edu>
Subject: SMTP e-mail test
To: #{email}

This is a test e-mail message.
"""
# construct a post request with email set to attack_string
# attack_string => sai@sai.com%0abcc:spc@spc.com%0aSubject:Hello
Net::SMTP.start('localhost', 1025) do |smtp|
smtp.send_message message, 'schand31@asu.edu',
'to@todomain.com'
end
# Headers get added, and Subject field changes to what we set.
end
\end{lstlisting}



