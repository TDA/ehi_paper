\subsection[Issues]{Design Issues}
\label{sys:issues}
This section will describe the issues we faced with the design decisions we made, and how we did our best to mitigate them, and their effect on the system.

\begin{itemize}

	\item Parallelism for the system\\
	\label{issues:parallel}
	Every component in the pipeline benefits hugely from parallel processing of the data. However, Python's GIL (Global Interpreter Lock) does not allow the running of multiple native threads concurrently. To overcome this, we used a Celery task queue (discussed in Section~\ref{exp:Celery}), which allowed a level of parallelism that Python does not provide by default. Even though this makes the system faster than a single-threaded approach, it still leaves room for improvement in terms of performance. Despite the speed drop that results from lack of full parallelism, we chose to go with Python, for the raw power it provides, its text processing capabilities, PCRE (Perl Compatible Regular Expressions) compatibility, and the numerous libraries available for parsing HTML, interfacing with databases and generating HTTP requests.

	\item URL Construction\\
	The multiple ways in which a URL is specified (i.e.\ Relative and Absolute URLs) complicates the construction of the URL from the \texttt{action} attribute of the form.  As an example, the following URLs are all equivalent (as parsed by a browser, assuming we are in the path \texttt{www.website.com}):

	\begin{itemize}
		\item \colorbox{lightgray}{\lstinline{action=mail.php}}
		\item \colorbox{lightgray}{\lstinline{action=./mail.php}}
		\item \colorbox{lightgray}{\lstinline{action=http://website.com/mail.php}}
		\item \colorbox{lightgray}{\lstinline{action=www.website.com/mail.php}}
	\end{itemize}
	Add to this, if the form is a self-referencing form\,\footnotemark, and is present in mail.php, the following are equivalent to the above URLs as well:
	\footnotetext{A self-referencing form is one which submits the form data to itself. It includes logic to both display the form and process it. It is a \emph{very} common feature in PHP-based scripts.}
	\begin{itemize}
		\item \colorbox{lightgray}{\lstinline{action=""}}
		\item \colorbox{lightgray}{\lstinline{action=#}}
		\item \texttt{action} is completely omitted
	\end{itemize}
	Also, relative URLs pose another problem. If the URL of the form page ends with `/' and the \texttt{action} specifies a path starting with `/' (illustrated in Listing~\ref{issues:url}), we would need to strip one of the two slashes. This increases the overall complexity of our URL generator, as we have to account for all these possibilities.

	\lstset{language=HTML,caption={URL construction, the resulting url needs to be www.website.com/mail.php and not www.website.com//mail.php},label={issues:url}}
	\begin{lstlisting}
	Current URL = www.website.com/
	<form action=/mail.php>
	\end{lstlisting}

	As using a browser engine to reconstruct these URLs  and connecting it to the fuzzer pipeline would have added unnecessary bulk to the project, we chose to go with a best-effort approach to this problem, where our system covers all these possibilities with a lightweight URL Generator, however, we cannot know for certain whether this works for other unforeseen ways of specifying a URL.

	\item Mapping responses to requests\\
	As we are generating multiple payloads for each form, and the received e-mail may not contain the name of the domain from which we received the e-mail, it is difficult to map the response e-mails to the right requests. We instead use the \texttt{form\_id} as part of the payload to map responses to requests accurately.

%	\item Crawling WordPress and other CMS-based websites\\
%	\label{issues:cms}

\end{itemize}