\subsection{Test Suite}
\label{Arch:Test}
The test plan for our system includes a set of unit tests for each module in the pipeline. Further, we have unit tests for every %every instead of each
individual function in the modules. The functions are tested separately, using mocks and stubs, so as to ensure isolated testing.
This section outlines the test plan in the following manner. We list the modules that are tested, and then describe what each unit test tests for.
\begin{itemize}
	\item Form Parser
	\begin{itemize}
		\item \colorbox{lightgray}{\lstinline{test_url_exception}} - Tests whether the system handles incorrect or malformed URLs properly and terminates cleanly.
		\item \colorbox{lightgray}{\lstinline{test_db_connection}} - Tests whether the Database Connection is set up and queries can be executed.
		\item \colorbox{lightgray}{\lstinline{test_form_parser}} - Tests for the proper parsing of HTML, and if the system exits cleanly in case parsing is not possible.
	\end{itemize}
	
	\item E-Mail Field Checker
	\begin{itemize}
		\item \colorbox{lightgray}{\lstinline{test_check_for_email}} - Tests whether the system finds E-Mail fields in the form when the words `e-mail' or `email' are present in the form (case insensitive).
		\item \colorbox{lightgray}{\lstinline{test_check_for_no_email}} - Tests whether the system finds no E-Mail fields when the words `e-mail' or `email' are \emph{not} present in the form (case insensitive).
	\end{itemize}
	
	\item E-Mail Form Retriever
	\begin{itemize}
		\item \colorbox{lightgray}{\lstinline{test_reconstruct_form}} - Tests for the proper reconstruction of the form stored in the Database.
		\item \colorbox{lightgray}{\lstinline{test_construct_url}} - Tests whether the URL for submission was constructed properly, includes checks for relative URLs, absolute URLs, and presence of \lstinline{`base'} tags.
		\item \colorbox{lightgray}{\lstinline{test_email_form_retriever_already_fuzzed}} - Tests for duplicate fuzz requests, and whether the system rejects these requests.
		\item \colorbox{lightgray}{\lstinline{test_email_form_retriever_calls_fuzzer_for_new_fuzz}} - Tests whether the E-Mail Form Retriever calls the Fuzzer module with the proper data when it gets a new fuzz request.
	\end{itemize}
	
	\item Fuzzer
	\begin{itemize}
		\item \colorbox{lightgray}{\lstinline{test_send_get_request}} - Tests for the proper handling of GET requests.
		\item \colorbox{lightgray}{\lstinline{test_send_post_request}} - Tests for the proper handling of POST requests.
		\item \colorbox{lightgray}{\lstinline{test_correct_fuzzer_data}} - Tests whether the payload generated for the given form data is correct and consistent. Also tests whether the payload was part of the resulting HTTP request.
		\item \colorbox{lightgray}{\lstinline{test_incorrect_fuzzer_data}} - Tests for incorrect form data, and ensures that a payload does not end up in the wrong input field in the resulting HTTP request.
	\end{itemize}
	\item E-Mail Analyzer
	\begin{itemize}
		\item \colorbox{lightgray}{\lstinline{test_load_mail}} - Tests whether the E-Mails are loaded and parsed correctly by the E-Mail Analyzer.
		\item \colorbox{lightgray}{\lstinline{test_parse_headers}} - Tests for the proper parsing of headers present in the E-Mail.
		\item \colorbox{lightgray}{\lstinline{test_analyze_regular_mail}} - Tests whether the E-Mail Analyzer parses the regular E-Mail properly and extracts the injected input fields that are present in the E-Mail.
		\item \colorbox{lightgray}{\lstinline{test_analyze_malicious_mail}} - Tests whether the E-Mail Analyzer parses the E-Mails received due to the malicious payloads properly, is able to extract the `bcc' headers, and is able to link them to the proper fuzzing request and payload.
		\item \colorbox{lightgray}{\lstinline{test_analyze_x_check_header}} - Tests whether the `x-check' header is read by the E-Mail Analyzer.
	\end{itemize}
\end{itemize}
The unit tests were written using Python's built-in `Unittest' module, mocking was done using the built-in `MagicMock' module. The tests allow us to be reasonably certain that our system works as expected.
