\subsection{System Architecture}
\label{sys:arch}
The black-box testing system can be divided broadly into two modules; Data Gathering and Payload Injection.
\begin{enumerate}
	\item Data Gathering\\
	The Data Gathering module is primarily responsible for the following activities:
	\begin{itemize}
		\item Interface with the Crawler (Section~\ref{Comp:Crawler}) and receive the URLs.
		\item Parse the HTML for the corresponding URL and store the relevant form data (Section~\ref{Comp:FP}).
		\item Check for the presence of forms that allow the user to send/receive e-mail, and store references to these forms (Section~\ref{Comp:EMFC}).
	\end{itemize} 
	\item Payload Injection\\
	The Payload Injection module is primarily responsible for the following activities:
	\begin{itemize}
		\item Retrieve the forms that allow users of a website to send/receive e-mail and reconstruct these forms (Section~\ref{Comp:EMFR}).
		\item Inject these forms with benign data (non-malicious payloads) and generate an HTTP request to the corresponding URL (Section~\ref{Comp:Fuzzer:nmp}).
		\item Analyze the e-mails, extracting the header fields and checking for the presence of the injected payloads (Section~\ref{Comp:EMA}).
		\item Inject the forms that sent us e-mails with malicious payloads, and generate an HTTP request to the corresponding URL to check if E-Mail Header Injection vulnerability exists in that form (Section~\ref{Comp:Fuzzer:mp}).
	\end{itemize} 
	The functionality of each component is discussed further in the `Components' section (Section~\ref{Comp}). The Payload Injection pipeline is not a linear, but cyclic process, as we inject different payloads and analyze the received e-mails.
\end{enumerate}
