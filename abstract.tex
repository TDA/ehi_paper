\begin{abstract}
	E-Mail header injection vulnerability is a class of vulnerability that can occur in web applications that use user input to construct e-mail messages. E-Mail header injection is possible when the mailing script fails to check for the presence of e-mail headers in user input (either form fields or URL parameters). The vulnerability exists in the reference implementation of the built-in mail functionality in popular languages such as PHP, Java, Python, and Ruby. With the proper injection string, this vulnerability can be exploited to inject additional headers, modify existing headers, and  alter the content of the e-mail.
	
	This paper presents a scalable mechanism to automatically detect E-Mail Header Injection vulnerabilities and uses this mechanism to quantify the prevalence of E\-Mail Header Injection vulnerabilities on the web. Using a black-box testing approach, the system crawled \urls URLs identify web pages which contained form fields. \forms such forms were found by the system, of which \emailforms forms contained e-mail fields. We then test \fuzzed forms to see if they would send us an e-mail, and \recd forms set us an email. Of these \malfuzzed forms were tested with E\-mail Header Injection payloads and, of these, we found \success vulnerable URLs across \domains domains. Then, to demonstrate that E\-mail Header Injection vulnerabilities are actively being exploited to create a spamming platform, we found \ipsblacklist IPs that were vulnerable on spaming blacklists. This work shows that E\-mail Header Injection vulnerabilities are widespread widespread and deserve future research attention.
\end{abstract}
