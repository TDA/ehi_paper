\section{Conclusions}
We have showcased a novel approach involving black-box testing to identify the presence of E-Mail Header Injection in a web application. Using this approach, we have demonstrated that our system was able to crawl \urls\ web pages finding \forms\ forms, out of which \emailforms\ forms were fuzzable. We fuzzed \fuzzed\ forms and found \recd\  forms that allowed us to send/receive e-mails. Out of these, we were able to inject malicious payloads into \malfuzzed\ forms, identifying \success\ vulnerable forms (\successDelta\ success rate). This indicates that the vulnerability is widespread, and needs attention from both web application developers and library makers. 

We hope that our work sheds light on the prevalence of this vulnerability and that it ensures that the implementation of the \texttt{mail} function in popular languages is fixed to differentiate between User-supplied headers, and headers that are legitimately added by the application, and that the RFC's are updated to be more stringent and make it less ambiguous for future implementations. 