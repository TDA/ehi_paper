\section[Limitations]{Limitations of the Project}
\label{limitations}
	This section complements Section \ref{sys:issues}, and discusses the limitations of our project. The following list, although not exhaustive, goes into the limitations of our project in detail: 
	\begin{itemize}
		\item CAPTCHAs - As noted in section \ref*{issues:captcha}, CAPTCHAs pose a significant problem to our automated system. As CAPTCHAs are designed to be robust, there is no easy way to break them. There has been considerable research in this area \cite{captchas2}, \cite{captchas} to name a few and although not impossible to break, it remains out of the scope of this project, and thus, we chose to ignore websites which require CAPTCHA verification.
		\item JavaScript Apps - Due to the growing emphasis on responsive web applications, more and more single-page web applications are being built purely with JavaScript. Even conventional applications are now making use of JavaScript to dynamically insert content and update the pages. This means that these dynamically injected components are not a part of the initial source code that is sent to the client by the web server.
		
		Thus, our system never receives dynamically injected forms from the web server and hence is unable to detect whether these vulnerabilities are present in such forms. The only workaround would be, to use a JavaScript engine to query for the \lstinline|document.getElementsByTagName('html')[0].innerHTML| (from inside web browser automation tools like Selenium, etc.), and then use that as the source code for our URL.
		
		Since this would add unnecessary bulk and complexity to our application, we chose not to do it, and thus, we consider this to be a limitation.
		
		\item Blogs powered by WordPress/Drupal\\
        In addition to what was discussed in Section \ref{issues:cms}, we found that certain WordPress `plugins' also prevent the E-Mail Header Injection attack by sanitizing user input on Contact Forms. Some of these plugins are discussed in the following section. Although not all websites built with WordPress are secure from the attack, between the presence of the plugins on some websites, and getting tagged as `spambots' by others, we were able to do vulnerability analysis on very few sites powered by WordPress.
		
		\item Blacklisting by websites and ISPs\\
        During the actual crawl, our system was blacklisted by a few websites (mostly WordPress ones), and Internet Service Providers (ISPs). We then created a blacklist of our own to ensure that we did not inject these websites. The result was that we could not gather any data about these websites.
		
		\item E-Mail libraries\\
        E-Mail libraries like the PHP Extension and Application Repository's (PEAR) mail library provide sanitization checks for user input. While this is technically not a limitation of our project, it still makes it such that we are not able to inject these sites successfully.
        A few other libraries for each language are discussed in the following section.
        
        \item Websites that are not in English\\
        Because we are only searching for the words `e-mail', `mail' or `email' within the form, if the website does not use English names for its forms, our system will not be able to find the presence of an e-mail field. An example is shown in Listing~\ref{code:htmlfrench}. Here, the French word for `e-mail' --- courrier électronique --- is used, and our system is unable to find the presence of the e-mail form.
	\end{itemize}
	
\lstset{language=HTML,caption={HTML page with e-mail form, written in a different language - French.},label={code:htmlfrench}, literate=%
	{é}{{\'e}}1}
\begin{lstlisting}
<!doctype html>
<html lang="fr">
<head>
<meta charset="utf-8">
<meta name="author" content="Sai Pc">
<title>Mock Email</title></head>
<body>
<form action="{Replace with path to back-end}" method="post">
<input type="text" placeholder="courrier électronique" 
	name="courrier_électronique"><br>
<textarea name="msg" rows="20" cols="120"></textarea>
<input type="submit" value="courrier électronique!">
</form></body>
</html>
\end{lstlisting}
