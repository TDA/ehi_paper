% This is "sig-alternate.tex" V2.1 April 2013
% This file should be compiled with V2.5 of "sig-alternate.cls" May 2012
%
% This example file demonstrates the use of the 'sig-alternate.cls'
% V2.5 LaTeX2e document class file. It is for those submitting
% articles to ACM Conference Proceedings WHO DO NOT WISH TO
% STRICTLY ADHERE TO THE SIGS (PUBS-BOARD-ENDORSED) STYLE.
% The 'sig-alternate.cls' file will produce a similar-looking,
% albeit, 'tighter' paper resulting in, invariably, fewer pages.
%
% ----------------------------------------------------------------------------------------------------------------
% This .tex file (and associated .cls V2.5) produces:
%       1) The Permission Statement
%       2) The Conference (location) Info information
%       3) The Copyright Line with ACM data
%       4) NO page numbers
%
% as against the acm_proc_article-sp.cls file which
% DOES NOT produce 1) thru' 3) above.
%
% Using 'sig-alternate.cls' you have control, however, from within
% the source .tex file, over both the CopyrightYear
% (defaulted to 200X) and the ACM Copyright Data
% (defaulted to X-XXXXX-XX-X/XX/XX).
% e.g.
% \CopyrightYear{2007} will cause 2007 to appear in the copyright line.
% \crdata{0-12345-67-8/90/12} will cause 0-12345-67-8/90/12 to appear in the copyright line.
%
% ---------------------------------------------------------------------------------------------------------------
% This .tex source is an example which *does* use
% the .bib file (from which the .bbl file % is produced).
% REMEMBER HOWEVER: After having produced the .bbl file,
% and prior to final submission, you *NEED* to 'insert'
% your .bbl file into your source .tex file so as to provide
% ONE 'self-contained' source file.
%
% ================= IF YOU HAVE QUESTIONS =======================
% Questions regarding the SIGS styles, SIGS policies and
% procedures, Conferences etc. should be sent to
% Adrienne Griscti (griscti@acm.org)
%
% Technical questions _only_ to
% Gerald Murray (murray@hq.acm.org)
% ===============================================================
%
% For tracking purposes - this is V2.0 - May 2012

\documentclass{sig-alternate-05-2015}
\usepackage[T1]{fontenc}
\usepackage[square,numbers]{natbib}
\usepackage[justification=centering]{caption}
\captionsetup[table]{skip=1pt}
\usepackage[pageanchor=true,plainpages=false,pdfpagelabels,bookmarks,bookmarksnumbered,hidelinks]{hyperref}
\usepackage{color}
\definecolor{mygray}{rgb}{0.6,0.6,0.6}
\definecolor{lightgray}{rgb}{0.92,0.92,0.92}
\definecolor{darkgreen}{rgb}{0,0.7,0}

% Great package for spaces in commands
\usepackage{xspace}

% for footnotes in tables
\usepackage{tablefootnote}

% for title caps
\usepackage{titlecaps}
\Addlcwords{and, the, or, of, that, our, by, a, prevent, for}

% default options for listings
\usepackage{listings}
\usepackage{listingsutf8}
\usepackage{textcomp}     % access \textquotesingle
\lstset{
	backgroundcolor=\color{lightgray},
	basicstyle=\scriptsize{}\ttfamily,
	breaklines=true, 
	captionpos=b,
	commentstyle=\color{darkgreen}, 
	frame=single,
	keywordstyle=\color{blue}, 
	numbers=left,
	numbersep=5pt,
	numberstyle=\tiny\color{mygray},
	rulecolor=\color{black},
	showstringspaces=false,
	upquote=true,
    belowskip=-6ex,
}

% Important for the layout
\clubpenalty=10000
\widowpenalty=10000

	
\newcommand{\dq}[1]{``{#1}''}
%% THIS IS THE DATA FOR OUR THESIS, UPDATING HERE WILL UPDATE EVERYWHERE %%
\newcommand{\urls}{23,553,796\xspace}

\newcommand{\forms}{7,354,425\xspace}
\newcommand{\formsDelta}{31.22\%\xspace}

\newcommand{\emailforms}{1,228,774\xspace}
\newcommand{\emailformsDelta}{16.71\%\xspace}

\newcommand{\fuzzed}{1,012,530\xspace}
\newcommand{\fuzzedDelta}{82.40\%\xspace}

\newcommand{\recd}{74,192\xspace}
\newcommand{\recdDelta}{7.33\%\xspace}

\newcommand{\diffFoundFuzz}{9,682\xspace}
\newcommand{\malfuzzed}{64,510\xspace}
\newcommand{\malfuzzedDelta}{86.95\%\xspace}

\newcommand{\success}{994\xspace}
\newcommand{\successDelta}{1.54\%\xspace}
\newcommand{\successWebsitesDelta}{0.038\%\xspace} % calc domains/total_domains 414/1,085,365

\newcommand{\slowSelenium}{31.58\%\xspace}
\newcommand{\slowParse}{11.05\%\xspace}

\newcommand{\domains}{414\xspace}
\newcommand{\ips}{604\xspace}

\newcommand{\emailedDefaultmailbox}{113\xspace}
\newcommand{\responses}{21\xspace}
\newcommand{\confirmed}{15\xspace}


\newcommand{\ipsblacklist}{157\xspace}
\newcommand{\ipsblacklistmulti}{46\xspace}

\newcommand{\ehibcc}{583\xspace}
\newcommand{\ehixcheck}{493\xspace}
\newcommand{\ehito}{229\xspace}
\newcommand{\ehibccxcheck}{310\xspace}
\newcommand{\ehitoxcheck}{15\xspace}
\newcommand{\ehinuserxcheck}{239\xspace}
\newcommand{\ehiuniquenuserxcheck}{182\xspace}

% these refer to unique domains, not unique forms
\newcommand{\uniqueforms}{1,085,365\xspace}
\newcommand{\uniqueemailforms}{198,306\xspace}

\newcommand{\totalattachmentcount}{2,950\xspace}
\newcommand{\totalvirusattachmentcount}{265\xspace}
\newcommand{\totalvirusemails}{443\xspace}

\newcommand{\ehi}{E-\nobreak{}mail Header Injection\xspace}
\newcommand{\emails}{\email{}s\xspace}
\newcommand{\emailed}{\email{}ed\xspace}    
\newcommand{\Email}{E-\nobreak{}mail\xspace}        
\newcommand{\Emails}{\Email{}s\xspace}

\makeatletter
\def\@copyrightspace{\relax}
\makeatother
\begin{document}
	
	% Copyright
	\setcopyright{acmcopyright}
	%\setcopyright{acmlicensed}
	%\setcopyright{rightsretained}
	%\setcopyright{usgov}
	%\setcopyright{usgovmixed}
	%\setcopyright{cagov}
	%\setcopyright{cagovmixed}
	
	
	% DOI
	%\doi{10.475/123_4}
	
	% ISBN
	%\isbn{123-4567-24-567/08/06}
	
	%Conference
	%\conferenceinfo{PLDI '13}{June 16--19, 2013, Seattle, WA, USA}
	
	%\acmPrice{\$15.00}
	
	%
	% --- Author Metadata here ---
	%\conferenceinfo{WOODSTOCK}{'97 El Paso, Texas USA}
	%\CopyrightYear{2007} % Allows default copyright year (20XX) to be over-ridden - IF NEED BE.
	%\crdata{0-12345-67-8/90/01}  % Allows default copyright data (0-89791-88-6/97/05) to be over-ridden - IF NEED BE.
	% --- End of Author Metadata ---
	
	%\title{E-jection fraction - Tracking how your website pumps out
	%E-Mails}
    \title{Measuring E-Mail Header Injections on the World
    	Wide Web}
	%
	% You need the command \numberofauthors to handle the 'placement
	% and alignment' of the authors beneath the title.
	%
	% For aesthetic reasons, we recommend 'three authors at a time'
	% i.e. three 'name/affiliation blocks' be placed beneath the title.
	%
	% NOTE: You are NOT restricted in how many 'rows' of
	% "name/affiliations" may appear. We just ask that you restrict
	% the number of 'columns' to three.
	%
	% Because of the available 'opening page real-estate'
	% we ask you to refrain from putting more than six authors
	% (two rows with three columns) beneath the article title.
	% More than six makes the first-page appear very cluttered indeed.
	%
	% Use the \alignauthor commands to handle the names
	% and affiliations for an 'aesthetic maximum' of six authors.
	% Add names, affiliations, addresses for
	% the seventh etc. author(s) as the argument for the
	% \additionalauthors command.
	% These 'additional authors' will be output/set for you
	% without further effort on your part as the last section in
	% the body of your article BEFORE References or any Appendices.
	
	\numberofauthors{1} %  in this sample file, there are a *total*
	% of EIGHT authors. SIX appear on the 'first-page' (for formatting
	% reasons) and the remaining two appear in the \additionalauthors section.
	%

    % Note: For CCS submission we need to have it anonymous
    % The author list will be:
    % Sai, Pierre-Marie, Chris, Giovanni, Ziming, Adam, Gail
    
	\author{
		% You can go ahead and credit any number of authors here,
		% e.g. one 'row of three' or two rows (consisting of one row of three
		% and a second row of one, two or three).
		%
		% The command \alignauthor (no curly braces needed) should
		% precede each author name, affiliation/snail-mail address and
		% e-mail address. Additionally, tag each line of
		% affiliation/address with \affaddr, and tag the
		% e-mail address with \email.
		%
		% 1st. author
		\alignauthor
		Sai Prashanth Chandramouli$^\dag$, Pierre-Marie Bajan$^\S$, Christopher Kruegel$^\ddag$, Giovanni Vigna$^\ddag$, Ziming Zhao$^\dag$, Adam Doup\'e$^\dag$, and Gail-Joon Ahn$^\dag$\\
		\affaddr{Arizona State University$^\dag$, IRT SystemX$^\S$, and University of California, Santa Barbara$^\ddag$}\\
		\{saipc, zzhao30, doupe, gahn\}@asu.edu$^\dag$,
        pierre-marie.bajan@irt-systemx.fr$^\S$, \\
        \{chris, vigna\}@cs.ucsb.edu$^\ddag$
        \vspace{-5ex}
        }
		% 2nd. author
		%% \alignauthor

		%% \affaddr{University of California, Santa Barbara}\\
		%% pm.bajan@free.fr
		%% \alignauthor
		%% \\
		%% \affaddr{University of California, Santa Barbara}\\
		%% giovanni.vigna@gmail.com
		%% \and
		%% \alignauthor
		%% \\
		%% \affaddr{Arizona State University}\\
		%% Ziming.Zhao@asu.edu
		%% \alignauthor
		%% Adam Doup\'{e}\\
		%% \affaddr{Arizona State University}\\
		%% doupe@asu.edu
		%% \alignauthor
		%% Gail-Joon Ahn\\
		%% \affaddr{Arizona State University}\\
		%% Gail-Joon.Ahn@asu.edu
		% 3rd. author
		%% \alignauthor Lars Th{\o}rv{\"a}ld\titlenote{This author is the
		%% 	one who did all the really hard work.}\\
		%% \affaddr{The Th{\o}rv{\"a}ld Group}\\
		%% \affaddr{1 Th{\o}rv{\"a}ld Circle}\\
		%% \affaddr{Hekla, Iceland}\\
		%% \email{larst@affiliation.org}
		%% \and  % use '\and' if you need 'another row' of author names
		%% % 4th. author
		%% \alignauthor Lawrence P. Leipuner\\
		%% \affaddr{Brookhaven Laboratories}\\
		%% \affaddr{Brookhaven National Lab}\\
		%% \affaddr{P.O. Box 5000}\\
		%% \email{lleipuner@researchlabs.org}
		%% % 5th. author
		%% \alignauthor Sean Fogarty\\
		%% \affaddr{NASA Ames Research Center}\\
		%% \affaddr{Moffett Field}\\
		%% \affaddr{California 94035}\\
		%% \email{fogartys@amesres.org}
		%% % 6th. author
		%% \alignauthor Charles Palmer\\
		%% \affaddr{Palmer Research Laboratories}\\
		%% \affaddr{8600 Datapoint Drive}\\
		%% \affaddr{San Antonio, Texas 78229}\\
		%% \email{cpalmer@prl.com}
	% There's nothing stopping you putting the seventh, eighth, etc.
	% author on the opening page (as the 'third row') but we ask,
	% for aesthetic reasons that you place these 'additional authors'
	% in the \additional authors block, viz.
	%% \additionalauthors{Additional authors: John Smith (The Th{\o}rv{\"a}ld Group,
	%% 	email: {\texttt{jsmith@affiliation.org}}) and Julius P.~Kumquat
	%% 	(The Kumquat Consortium, email: {\texttt{jpkumquat@consortium.net}}).}
	%% \date{30 July 1999}
	% Just remember to make sure that the TOTAL number of authors
	% is the number that will appear on the first page PLUS the
	% number that will appear in the \additionalauthors section.
	\renewcommand{\email}{e-\nobreak{}mail\xspace}
	\maketitle


	\begin{abstract}
	E-Mail header injection vulnerability is a class of vulnerability that can occur in web applications that use user input to construct e-mail messages. E-Mail injection is possible when the mailing script fails to check for the presence of e-mail headers in user input (either form fields or URL parameters). The vulnerability exists in the reference implementation of the built-in \dq{mail} functionality in popular languages like PHP, Java, Python, and Ruby\@. With the proper injection string, this vulnerability can be exploited to inject additional headers and/or modify existing headers in an E-Mail message, allowing an attacker to completely alter the content of the e-mail.
	\paragraph{}
	This thesis develops a scalable mechanism to automatically detect E-Mail Header Injection vulnerability and uses this mechanism to quantify the prevalence of E\-Mail Header Injection vulnerabilities on the Internet. Using a black-box testing approach, the system crawled \urls\ URLs to find URLs which contained form fields. \forms\ such forms were found by the system, of which \emailforms\ forms contained e-mail fields. The system used this data feed to discern the forms that could be fuzzed with malicious payloads. Amongst the \fuzzed\ forms tested, \recd\ forms were found to be injectable with more malicious payloads. The system tested \malfuzzed\ of these, and was able to find \success\ vulnerable URLs across \domains\ domains, which proves that the threat is widespread and deserves future research attention.

\end{abstract}

	
	
	%
	% The code below should be generated by the tool at
	% http://dl.acm.org/ccs.cfm
	% Please copy and paste the code instead of the example below. 
	%
	%% \begin{CCSXML}
	%% 	<ccs2012>
	%% 	<concept>
	%% 	<concept_id>10010520.10010553.10010562</concept_id>
	%% 	<concept_desc>Computer systems organization~Embedded systems</concept_desc>
	%% 	<concept_significance>500</concept_significance>
	%% 	</concept>
	%% 	<concept>
	%% 	<concept_id>10010520.10010575.10010755</concept_id>
	%% 	<concept_desc>Computer systems organization~Redundancy</concept_desc>
	%% 	<concept_significance>300</concept_significance>
	%% 	</concept>
	%% 	<concept>
	%% 	<concept_id>10010520.10010553.10010554</concept_id>
	%% 	<concept_desc>Computer systems organization~Robotics</concept_desc>
	%% 	<concept_significance>100</concept_significance>
	%% 	</concept>
	%% 	<concept>
	%% 	<concept_id>10003033.10003083.10003095</concept_id>
	%% 	<concept_desc>Networks~Network reliability</concept_desc>
	%% 	<concept_significance>100</concept_significance>
	%% 	</concept>
	%% 	</ccs2012>  
	%% \end{CCSXML}
	
	%% \ccsdesc[500]{Computer systems organization~Embedded systems}
	%% \ccsdesc[300]{Computer systems organization~Redundancy}
	%% \ccsdesc{Computer systems organization~Robotics}
	%% \ccsdesc[100]{Networks~Network reliability}
	
	
	%
	% End generated code
	%
	
	%
	%  Use this command to print the description
	%
	% \printccsdesc
	
	% We no longer use \terms command
	%\terms{Theory}
	
	% \keywords{ACM proceedings; \LaTeX; text tagging}

	\chapter{Introduction}
\paragraph{}
	The World Wide Web has single-handedly brought about a change in the way we use computers. The ubiquitous nature of the Web has made it possible for the general public to access it anywhere and on multiple devices like phones, laptops, personal digital assistants, and even on TVs and cars. This has ushered in an era of responsive web applications which depend on user input. While this rapid pace of development has improved the speed of dissemination of information, it does come at a cost. Attackers have an added incentive to break into user's e-mail accounts more than ever. E-Mail accounts are usually connected to almost all other online accounts of a user, and e-mails continue to serve as the principal mode of official communication on the web for most institutions. Thus, the impact an attacker can have by having control over the e-mail communication sent by websites to users is of an enormous magnitude. 
	
	Since attackers typically masquerade themselves as users of the system, if user input is to be trusted, then developers need to have proper sanitization routines in place. Many different injection attacks such as SQL injection or cross-site scripting (XSS) \cite{OWASPT10} are possible due to improper sanitization of user input. 
	
	Our research focuses on a lesser known injection attack known as E-Mail Header Injection. E-Mail Header Injection can be considered as the e-mail equivalent of HTTP Header Injection vulnerability \cite{wiki:HTTP_headerinjection}. The vulnerability exists in the reference implementation of the built-in \dq{\texttt{mail}} functionality in popular languages like PHP, Java, Python, and Ruby. With the proper injection string, this vulnerability can be exploited to inject additional headers and/or modify existing headers in an e-mail message --- with the potential to alter the contents of the e-mail message --- while still appearing to be from a legitimate source.
	
	E-Mail Header Injection attacks have the potential to allow an attacker to perform e-mail spoofing, resulting in phishing attacks that can lead to identity theft.
	The objective of our research is to study the prevalence of this vulnerability on the World Wide Web, and identify whether further research is required in this area.
	
	We performed an expansive crawl of the web, extracting forms with e-mail fields, and injecting them with different payloads to infer the existence of E-Mail Header Injection vulnerability. We then audited received e-mails to see if any of the injected data was present. This allowed us to classify whether a particular URL was vulnerable to the attack. The entire system works in a black-box manner, without looking at the web application's source code, and only analyzes the e-mails we receive based on the injected payloads.

\paragraph{Structure of document} % describes the remaining sections and gives a short desc about them
This thesis document is divided logically into the following sections:
\begin{itemize}
	\item Chapter 2 discusses the background of E-Mail Header Injection, a brief history of the vulnerability, and enumerates the languages and platforms affected by this vulnerability.
	
	\item Chapter 3 discusses the System design, the architecture, and the components of the system.
	
	\item Chapter 4 describes the experimental setup and sheds light on how we overcame the issues and assumptions discussed in Chapter 3.
	
	\item Chapter 5 presents our findings and our analysis of the results.
	
	\item Chapter 6 continues the discussion of the results; the lessons learned over the course of the project, limitations, and a suitable mitigation strategy to overcome the vulnerability.
	
	\item Chapter 7 explores related work in the area.
	
	\item Chapter 8 concludes this thesis, with ideas to expand the research in this area.
\end{itemize} 

\paragraph{} % summary paragraph
We hope that our research sheds some light on this relatively less well-known vulnerability, and find out its prevalence on the World Wide Web. In summary, we make the following contributions:
\begin{itemize}
	
	\item{A black-box approach to detecting the presence of E-Mail Header Injection vulnerability in a web application.}
	
	\item{A detection and classification tool based on the above approach, which will automatically detect such E-Mail Header Injection vulnerabilities in a web application.}
	
	\item{A quantification of the presence of such vulnerabilities on the World Wide Web, based on a crawl of the Web, including \urls\ URLs and \forms\ forms.}
	
\end{itemize}

\paragraph{}
	
	\section{Background}
\ehi belongs to a broad class of vulnerabilities known as command injection vulnerabilities. However, unlike its more popular siblings, SQL injection~\cite{sql1, sql0, sql2}, Cross-Site Scripting~\cite{Injection1, KleinAmit}, or HTTP Header Injection~\cite{sessionride}, relatively little research is available on \ehi vulnerabilities.

As with other vulnerabilities in this class, \ehi is caused due to improper or nonexistent sanitization of user input. If the program that constructs \emails from user input fails to check for the presence of \email headers in the user input, a malicious user---using a well-crafted payload---can control the headers set for this particular e-mail. \ehi vulnerabilities can be leveraged to enable malicious attacks, including, but not limited to, spoofing or phishing.

\subsection{History of \ehi}
\begin{lstlisting}[language=PHP,caption={PHP program with e-mail
      header injection vulnerability.},label={code:phpemi}, float]
$from = $_REQUEST['email'];
$subject = 'Hello XYZ';
$message = 'We need you to reset your password';
$to = 'xyz@example.com';
// example attack string injected as for
// $_REQUEST['email'] => 
// 'abc@example.com\nCC:spc@example.com'
$retValue = mail($to, $subject, $message, "From: $from");
// E-Mail gets sent to both 
// xyz@example.com AND spc@example.com
\end{lstlisting}

We found the first \ehi description in a late 2004 article on phpsecure.info~\cite{Tobozo} accredited to user \lstinline|tobozo| describing how an \ehi vulnerability existed in the implementation of the \texttt{mail()} function in PHP and how it can be exploited. More recently, a blog post by Damon Kohler~\cite{DK} and an accompanying wiki article~\cite{Injection} describe the attack vector and outline few defense measures for \ehi vulnerabilities.

%% As this vulnerability was initially found in the \texttt{mail()} function of PHP, \ehi can be traced to as early as the beginning of the 2000's, present in the \texttt{mail()} implementation of PHP 4.0. 

An example of the vulnerable code written in PHP is shown in Listing~\ref{code:phpemi}. This code takes in user input from the PHP superglobal \texttt{\$\_REQUEST[\textquotesingle email\textquotesingle]}, and stores it in the variable \texttt{\$from}, which is later passed to the \texttt{mail()} function to construct and send the e-mail.


\lstset{language=HTML,caption={SMTP headers generated by a PHP mailing script.},label={code:smtpheaders}}
\begin{lstlisting}
Received: from mail.ourdomain.com ([62.121.130.29])
	by xyz.com (Postfix) with ESMTP id 5A08E52C0154
	for <s@s.com>; Sun, 20 Mar 2016 13:56:58 -0700 (MST)
To: s@s.com
Subject: Hello XYZ
CC: spc@spc.com
Date: Sun, 20 Mar 2016 13:56:58 -0700 (MST)

We need you to reset your password
\end{lstlisting}
\begin{sloppypar}
When this code is given the malicious input \texttt{\lstinline{abc@example.com\\nCC:spc@example.com}} as the value of the \texttt{\$\_REQUEST[\textquotesingle email\textquotesingle]}, it generates the equivalent SMTP Headers shown in Listing~\ref{code:smtpheaders}. It can be seen that the \texttt{CC} (carbon copy) header that we injected appears as part of the resulting SMTP message. This will make the e-mail get sent to the e-mail address specified as part of the \texttt{CC} as well. 

%\begin{table}[tbp]
  \scriptsize
	\centering
	\scalebox{0.85}{
	\begin{tabular}{|c|p{5.5cm}|c|}
		\hline
		\multicolumn{1}{|c|}{\textbf{CVE No.}} & 
		\multicolumn{1}{c|}{\textbf{Affected Software}} &
		\multicolumn{1}{c|}{\textbf{Year}}\\
		\hline
		{2002-1575} & {cgiemail} & {2004}\\
		\hline
		{2002-1771} & {FormMail 1.9} & {2005}\\
		\hline
		{2002-1917} & {Geeklog 1.35} & {2005}\\
		\hline
		{2005-0493} & {Biz Mail Form <=2.2} & {2005}\\
		\hline
		{2005-2854} & {thesitewizard.com} & {2005}\\
		\hline
		{2005-3883} & {PHP mb\_send\_mail} & {2005}\\
		\hline
		{2006-0631} & {Perl mailback.pl} & {2006}\\
		\hline
		{2006-0712} & {Squishdot 1.5.0} & {2006}\\
		\hline
		{2006-1225} & {Drupal 4.5.0-4.5.8 and 4.6.0-4.6.8} & {2006}\\
		\hline
		{2006-1305} & {Microsoft Outlook 2000, 2002-03} & {2006}\\
		\hline
		{2006-2159} & {Russcom Network} & {2006}\\
		\hline
		{2006-2943} & {CGI-RESCUE WebFORM 4.1} & {2006}\\
		\hline
		{2006-2944} & {CGI-RESCUE FORM2MAIL 1.21} & {2006}\\
		\hline
		{2006-3171} & {CS-Forum <=0.82} & {2006}\\
		\hline
		{2006-3473} & {Drupal Module <=1.8.2.2} & {2006}\\
		\hline
		{2006-4344} & {CGI-Rescue Mail} & {2006}\\
		\hline
		{2006-7020} & {phpwcms 1.2.5-DEV} & {2007}\\
		\hline
		{2006-7087} & {Dotdeb PHP} & {2007}\\
		\hline
		{2007-1718} & {PHP 4.0-4.4.6 and 5.0-5.2.1} & {2007}\\
		\hline
		{2007-1898} & {Jetbox CMS 2.1} & {2007}\\
		\hline
		{2007-1900} & {FILTER\_VALIDATE\_EMAIL PHP} & {2007}\\
		\hline
		{2007-2731} & {Jetbox CMS 2.1} & {2007}\\
		\hline
		{2008-2105} & {Bugzilla} & {2008}\\
		\hline
		{2009-1469} & {IceWarp} & {2009}\\
		\hline
		{2008-7281} & {OTRS - Open Ticket Request System} & {2011}\\
		\hline
		{2014-2957} & {Exim} & {2014}\\
		\hline
		{2015-8476} & {PHPMailer} & {2015}\\
		\hline
		{2016-4803} & {dotCMS} & {2016}\\
		\hline

	\end{tabular}
	}
	\caption{History of software found in Common Vulnerabilities and
      Exposures database affected by e-mail header injection
      vulnerability}
    \vspace{-5ex}
	\label{tab:history}
\end{table}
%TODO Adam: added this table

\end{sloppypar}

%TODO Adam: added this para abt E-Mail Header Injection history from CVE
We gathered reported Common Vulnerabilities and Exposures
(CVE)~\cite{cve} to get an idea of the distribution of
reported \ehi vulnerabilities over time. From the 28 reports we found (Table~\ref*{tab:history}), it can be seen that even though many vulnerabilities were found in earlier years (2005-07), there have been recently discovered \ehi vulnerabilities which suggests that it is still a very real and relevant threat to modern web security.

\begin{table}[tbp]
  \scriptsize
	\centering
	\scalebox{0.85}{
	\begin{tabular}{|c|p{5.5cm}|c|}
		\hline
		\multicolumn{1}{|c|}{\textbf{CVE No.}} & 
		\multicolumn{1}{c|}{\textbf{Affected Software}} &
		\multicolumn{1}{c|}{\textbf{Year}}\\
		\hline
		{2002-1575} & {cgiemail} & {2004}\\
		\hline
		{2002-1771} & {FormMail 1.9} & {2005}\\
		\hline
		{2002-1917} & {Geeklog 1.35} & {2005}\\
		\hline
		{2005-0493} & {Biz Mail Form <=2.2} & {2005}\\
		\hline
		{2005-2854} & {thesitewizard.com} & {2005}\\
		\hline
		{2005-3883} & {PHP mb\_send\_mail} & {2005}\\
		\hline
		{2006-0631} & {Perl mailback.pl} & {2006}\\
		\hline
		{2006-0712} & {Squishdot 1.5.0} & {2006}\\
		\hline
		{2006-1225} & {Drupal 4.5.0-4.5.8 and 4.6.0-4.6.8} & {2006}\\
		\hline
		{2006-1305} & {Microsoft Outlook 2000, 2002-03} & {2006}\\
		\hline
		{2006-2159} & {Russcom Network} & {2006}\\
		\hline
		{2006-2943} & {CGI-RESCUE WebFORM 4.1} & {2006}\\
		\hline
		{2006-2944} & {CGI-RESCUE FORM2MAIL 1.21} & {2006}\\
		\hline
		{2006-3171} & {CS-Forum <=0.82} & {2006}\\
		\hline
		{2006-3473} & {Drupal Module <=1.8.2.2} & {2006}\\
		\hline
		{2006-4344} & {CGI-Rescue Mail} & {2006}\\
		\hline
		{2006-7020} & {phpwcms 1.2.5-DEV} & {2007}\\
		\hline
		{2006-7087} & {Dotdeb PHP} & {2007}\\
		\hline
		{2007-1718} & {PHP 4.0-4.4.6 and 5.0-5.2.1} & {2007}\\
		\hline
		{2007-1898} & {Jetbox CMS 2.1} & {2007}\\
		\hline
		{2007-1900} & {FILTER\_VALIDATE\_EMAIL PHP} & {2007}\\
		\hline
		{2007-2731} & {Jetbox CMS 2.1} & {2007}\\
		\hline
		{2008-2105} & {Bugzilla} & {2008}\\
		\hline
		{2009-1469} & {IceWarp} & {2009}\\
		\hline
		{2008-7281} & {OTRS - Open Ticket Request System} & {2011}\\
		\hline
		{2014-2957} & {Exim} & {2014}\\
		\hline
		{2015-8476} & {PHPMailer} & {2015}\\
		\hline
		{2016-4803} & {dotCMS} & {2016}\\
		\hline

	\end{tabular}
	}
	\caption{History of software found in Common Vulnerabilities and
      Exposures database affected by e-mail header injection
      vulnerability}
    \vspace{-5ex}
	\label{tab:history}
\end{table}
%TODO Adam: added this table


\subsection{Languages Affected}
\label{languages}
We now describe the popular programming languages that contain \ehi vulnerabilities in their standard \email libraries. This section is not intended as a complete reference of vulnerable functions and methods, but rather as a guide that specifies which parts of the language are vulnerable.

\noindent{\textbf{PHP}} was one of the first languages found to be vulnerable to \ehi in its implementation of the \texttt{mail()} function at the time of release of PHP~4.0. The early finding of this vulnerability can be attributed in part to the success and popularity of the language for creating web pages. According to w3techs~\cite{W3techs}, PHP is used by 81.9\% of all the websites.

After 13 further iterations of PHP since the 4.0 release (the current version is 7.1), the \texttt{mail()} function is yet to be fixed after 15 years. However, it is specified in the PHP documentation~\cite{PHPDocs} that the \texttt{mail()} function does not protect against \ehi.
A working code sample of the vulnerability, written in PHP~5.6 is shown in  Listing~\ref{code:phpemi}.

\begin{sloppypar}
\noindent{\textbf{Python}}
A bug was filed about an \ehi vulnerability in Python's implementation of the \texttt{email.header} library and the header parsing functions allowing newlines in early 2009, which was followed by a partial patch in early 2011.
\end{sloppypar}

Unfortunately, the bug fix was only for the \texttt{email.header} package, and not for other frequently used packages such as \texttt{email.parser}, where both the classic \texttt{Parser()} and the newer \texttt{FeedParser()} contain \ehi vulnerabilities even in the latest versions: \texttt{2.7.11} and \texttt{3.5}. The bug fix was also not backported to older versions of Python.
There is no mention of the vulnerability in the Python documentation for either library. Contrary to PHP's behavior of overwriting existing headers, Python only recognizes the first occurrence of a header, and ignores duplicate headers.
A working code sample of the vulnerability, written in Python 2.7.11, is shown in Listing~\ref{code:pyemi}.

\begin{lstlisting}[language=Python,caption={Python program with e-mail
      header injection vulnerability.},label={code:pyemi}, float]
from email.parser import Parser
import cgi
form = cgi.FieldStorage()
to = form["email"]
msg = """To: """ + to + """\n
From: <user@example.com>\n
Subject: Test message\n\n
Body would go here\n"""

f = FeedParser() 
f.feed(msg)
headers = FeedParser.close(f)
print 'To: %s' % headers['to']
print 'BCC: %s' % headers['bcc']
\end{lstlisting}


\noindent{\textbf{Java}} has a bug report about \ehi filed against its \texttt{JavaMail} API. A detailed write-up by Alexandre Herzog~\cite{Herzog.2014} contains a proof-of-concept program that exploits the API to inject headers.

\begin{sloppypar}
\noindent{\textbf{Ruby}}
From our preliminary testing Ruby's built-in \texttt{Net::SMTP} library also has an \ehi vulnerability (not documented on the library's homepage).
%A working code sample of the vulnerability, written in Ruby 2.0.0 (the latest stable version at the time of writing), is shown in Listing~\ref{code:rubyemi}.
\end{sloppypar}
%\lstset{language=Ruby,caption={Ruby program with e-mail header injection vulnerability.},label={code:rubyemi}}
\begin{lstlisting}
require 'sinatra'
require 'net/smtp'

get '/hello' do
email = params[:email]

message = """
From: Sai <schand31@asu.edu>
Subject: SMTP e-mail test
To: #{email}

This is a test e-mail message.
"""
# construct a post request with email set to attack_string
# attack_string => sai@sai.com%0abcc:spc@spc.com%0aSubject:Hello
Net::SMTP.start('localhost', 1025) do |smtp|
smtp.send_message message, 'schand31@asu.edu',
'to@todomain.com'
end
# Headers get added, and Subject field changes to what we set.
end
\end{lstlisting}



\subsection{Exploitation}
\label{exploitation}
Successful exploitation of an \ehi vulnerability depends on where
injection occurs in the SMTP message. The attacker cannot alter parts
of the SMTP message that precede the injection location, but the
attacker has complete control over everything that follows. However,
similar to other command injection vulnerabilities, the remaining
parts of the SMTP message will always be appended to the attacker's
injection, so the attacker must contend with this. By
exploiting an \ehi vulnerability, an attacker can control who receives
the message (and can include multiple \texttt{CC} and \texttt{BCC}
recipients), the body, and possibly the subject (depending on if the subject header occurs
before/after the injection point and the language used).

\begin{lstlisting}[language=HTML,caption={Exploiting the \ehi
      vulnerability in Listing~\ref{code:phpemi} to control the
      recipients, subject, and body of the SMTP message.},label={code:ehiexploit}, float]
Received: from mail.ourdomain.com ([62.121.130.29])
  by xyz.com (Postfix) with ESMTP id 5A08E52C0154
  for <abc@example.com>; Sun, 20 Mar 2016 13:56:58 -0700 (MST)
From: abc@example.com
CC: 1@example.com, 2@example.com, 3@example.com
Subject: My Subject
Content-Type: multipart/mixed; boundary=foobar;
--foobar
Content-Type: text/html

This is the attacker's body
--foobar
To: xyz@example.com
Subject: Hello XYZ
Date: Sun, 20 Mar 2016 13:56:58 -0700(MST)

We need you to reset your password
\end{lstlisting}

The main vector for exploiting \ehi vulnerabilities follows the
template of command injection vulnerability exploitation: first inject
the attacker's desired commands, then comment out the rest of the
message. In \ehi vulnerabilities, the attacker first includes all SMTP
headers she desires. These will typically be the \texttt{Subject}
header to control the subject of the \email\footnotemark, \texttt{CC}
or \texttt{BCC} headers to control the recipients of the \email.

\footnotetext{The SMTP protocol
specifies that there should only be one \texttt{Subject} header, so
the attacker may not be able to alter the subject if the header is
already defined. This behavior would be MUA-dependent.
}

To handle the extra content after the injection point, one technique
is to use a \texttt{Content-type} header to specify that the SMTP
message is a multi-part email and that the sections are separated by
an attacker-specified boundary. The boundary delineates different
parts of the message so that the attacker's body is the only valid
part of the message, and the attacker can choose a random value for
the boundary that is not present in the developer-controlled part of
the SMTP message.

Using this technique, the attacker can completely control the \email.
For instance, injecting the following attack payload:
\texttt{\lstinline{abc@example.com\\nCC:1@example.com, 2@example.com,
    3@example.com\\nSubject: My
    Subject\\nContent-Type:multipart/mixed;
    boundary=foobar;\\n--foobar\\nContent-Type: text/html
    \\n\\nThis is the attacker's body\\n--foobar}} into the \texttt{email} parameter
of the PHP program in Listing~\ref{code:phpemi} results in
the SMTP message shown in Listing~\ref{code:ehiexploit}.

By expanding on this technique, the attacker can include links in the
\email, or even attachments, by adding additional multipart messages
with different content types.

A shorter technique, in case the injection point is limited in input
size, is to use an HTML comment to ignore the developer-controlled
part of the SMTP message, using a payload such as:
\texttt{\lstinline{abc@example.com\\nCC:1@example.com, 2@example.com,
    3@example.com\\nSubject: My Subject\\n
    Content-Type: text/html\\n\\nThis is the attacker's body<!--}}. However, this
technique will only work if the developer-controlled part of the SMTP
message does not contain a closing HTML comment tag
\texttt{\lstinline{-->}}.


\subsection{Impact of \ehi}

The impact of an \ehi vulnerability can be far-reaching. According to
w3tech, PHP, Java, Python, and Ruby (combined) account for over
85\%\, of the server-side programming languages in
websites measured, and the default implementation of the \email functionality of these languages is vulnerable to \ehi. 



%% \begin{table}[!tb]
%% 	\centering
%% 	\begin{tabular}{|p{4cm}|p{4cm}|}
%% 		\hline
%% 		\multicolumn{1}{|c|}{\textbf{Server Side Language}} & \multicolumn{1}{c|}{\textbf{\% of Usage}}\\
%% 		\hline
%% 		PHP & 81.9\\
%% 		\hline
%% 		Java & 3.1\\
%% 		\hline
%% 		Ruby & 0.6\\
%% 		\hline
%% 		Python & 0.2\\
%% 		\hline
		
%% 	\end{tabular}
%% 	\caption[\titlecap{Language usage statistics}]{Language usage statistics compiled from w3techs~\cite{W3techs}.}
%% 	\label{tab:usage}
%% \end{table}

An \ehi vulnerability can be exploited to do potentially any of the
following:

\noindent \textbf{Phishing and Spoofing Attacks} 
    Phishing~\cite{phishing} (a variation of spoofing~\cite{spoofing_attack}) refers to an attack where the recipient of an \email is made to believe that the \email is  legitimate when it was really created by a malicious party. The \email usually redirects the victim to a malicious website, which then steals their credentials or infects their computer with malware (via a drive-by-download).  
    
    \ehi gives attackers the ability to inject arbitrary headers into an \email sent by a website \emph{and control the output of the \email}. This adds credibility to the generated \email, as it is sent from the website's mail server and users (and anti-spam defenses) are more likely to trust an e-mail that is received from the proper mail server. Therefore, attackers could leverage \ehi vulnerabilities to perform enhanced phishing attacks. 
	
\noindent\textbf{Spam Networks}
	Spam networks can use \ehi vulnerabilities to send a large amount of \email from servers that are trusted. By adding additional \texttt{cc} or \texttt{bcc} headers to the generated e-mail, attackers can easily choose the recipient of the spam email. 
	
	Due to the \email being from trusted domains, recipient \email clients and anti-spam systems might not flag them as spam. If they do flag them as spam, then that can lead to the website being blacklisted as a spam generator (which would cause a Denial of Service on the vulnerable web application). 
	
\noindent\textbf{Information Extraction}
	\Emails can contain sensitive data that is meant to be accessed only by the user. Due to an \ehi vulnerability, an attacker can add a \texttt{bcc} header, and the \email server will send a copy of the \email to the attacker, thereby exposing important information.
	User privacy can thus be compromised, and loss of private information can by itself lead to other attacks.

    \noindent\textbf{Denial of Service}
    Denial of service attacks (DoS), can also be caused by exploiting an \ehi vulnerability to send excessive \emails resulting in overloading the mail server and cause crashes or instability. 

%It is evident that E-Mail Header Injection is a critical vulnerability that web applications must address.


	
	\section{System Design}
This section discusses the System design, and explains the architecture and the components of the System in detail. It then proceeds to enumerate the issues faced and the assumptions made during the building of the system. 
\subsection{Approach}
\label{sys:appr}
We took a black-box approach to measure the prevalence of \ehi vulnerability on the web. Black-box testing~\cite{Beizer:1995:BTT:202699} is a way to examine the functionality of an application without analyzing its source code.

Black-box testing allows our system to detect \ehi vulnerabilities in \emph{any} server-side language (not simply those we identified in Section~\ref{languages}).

The overall architecture of our system is presented in Figure~\ref{fig:overall}. The components shown in Figure~\ref{fig:overall} are discussed in the following section.

\begin{figure*}
	\centering
	\includegraphics[width=.8\textwidth]{overall}
	\caption{Overall system architecture.}
	\label{fig:overall}
\end{figure*}

\section{System Architecture}
\label{sys:arch}
The black-box testing system can be divided broadly into two modules; Data Gathering and Payload Injection.
\begin{enumerate}
	\item Data Gathering\\
	The Data Gathering module (shown in Figure~\ref{fig:crawler}) is primarily responsible for the following activities:
	\begin{itemize}
		\item Interface with the Crawler (Section \ref{Comp:Crawler}) and receive the URLs.
		\item Parse the HTML for the corresponding URL and store the relevant form data (Section \ref{Comp:FP}).
		\item Check for the presence of forms that allow the user to send/receive e-mail, and store references to these forms (Section \ref{Comp:EMFC}).
	\end{itemize} 
	\item Payload Injection\\
	The Payload Injection module (shown in Figure~\ref{fig:fuzzer}) is primarily responsible for the following activities:
	\begin{itemize}
		\item Retrieve the forms that allow users of a website to send/receive e-mail and reconstruct these forms (Section \ref{Comp:EMFR}).
		\item Inject these forms with benign data (non-malicious payloads) and generate an HTTP request to the corresponding URL (Section \ref{Comp:Fuzzer:nmp}).
		\item Analyze the e-mails, extracting the header fields and checking for the presence of the injected payloads (Section \ref{Comp:EMA}).
		\item Inject the forms that sent us e-mails with malicious payloads, and generate an HTTP request to the corresponding URL to check if E-Mail Header Injection vulnerability exists in that form (Section \ref{Comp:Fuzzer:mp}).
	\end{itemize} 
	The functionality of each component is discussed further in the `Components' section (Section~\ref{Comp}). The Payload Injection pipeline is not a linear, but cyclic process, as we inject different payloads and analyze the received e-mails.
\end{enumerate}

\begin{figure}
	\centering
	\includegraphics[width=14cm, height=7cm]{System/crawler_design}
	\caption[\titlecap{System architecture - crawler}]{System architecture - crawler \& form parser.}
	\label{fig:crawler}
\end{figure}


\begin{figure}
	\centering
	\includegraphics[width=16cm, height=9cm]{System/fuzzer_design}
	\caption[\titlecap{System architecture - fuzzer {\&} e-mail analyzer}]{System architecture - fuzzer {\&} e-mail analyzer.}
	\label{fig:fuzzer}
\end{figure}

\section{System Components}
\label{Comp}

The Data Gathering module and Payload Injection module are made up of a number of smaller components. This section describes in detail the functionality of each of the components.

\subsection{Crawler}
\label{Comp:Crawler}
We used an open-source Apache Nutch based Crawler. The Crawler provides us with a continuous feed of URLs and the HTML contained in those pages. This feed is sent to our Form Parser over a Celery Queue.

\subsection{Form Parser}
\label{Comp:FP}
The actual pipeline begins at the Form Parser. This module is responsible for parsing the HTML and retrieving data about the forms on the page, including the following:
\begin{itemize}
	\item Form attributes, such as method and action. These dictate where we send the HTTP request and what kind of request it is (GET or POST).
	\item Data about the input fields, such as their attributes, names, and default values. The default values are essential for fields like \colorbox{lightgray}{\lstinline{<input type="hidden">}} as these fields are usually used to check for the submission of forms by bots.
	\item Presence of the \colorbox{lightgray}{\lstinline{<base>}} element in the HTML, as this affects the final URL to which the form is to be submitted.
	\item Headers associated with the page, such as \emph{referrer}. Once again, these were required to avoid the website from ignoring our system as a bot.
\end{itemize} 
The Form Parser stores all this data in our database, so as to allow us to reconstruct the forms later for fuzzing, as required.

\subsection{E-Mail Field Checker}
\label{Comp:EMFC}
The E-Mail Field Checker script is the final stage in the Data Gathering pipeline. It receives the output of the previous stage---form data from the queue---and checks for the presence of e-mail fields in those forms. If any e-mail fields are found, it stores references to these forms in a separate table. This separates the forms that are potentially vulnerable from the forms that are not.

The E-Mail Field Checker searches for the words `e-mail', `mail' or `email' within the form, instead of an explicit e-mail field (e.g.,\ \colorbox{lightgray}{\lstinline{<input type="email">}}). This is by design, taking into account a very common design pattern used by web developers, where they may have a text field with an \dq{\texttt{id}} or \dq{\texttt{name}} set to `email', instead of an actual e-mail field, for purposes of backward compatibility with older browsers.

Compared to searching for explicit e-mail fields, by searching for the presence of the words `e-mail', `mail' or `email' in the form, we are assured very few false negatives. This is because our system is bound to find e-mail fields with their \dq{\texttt{type}}, \dq{\texttt{name}}, or \dq{\texttt{id}} set to one of these words. The system is also substantially faster as we do not have to parse the individual form fields at this point in the pipeline. However, despite the advantages, this might also lead to a false positive rate. We discuss this possibility in detail in Section \ref*{issues:fpr} - Design Issues. 

The output of this stage is stored in the database for persistence and acts as the input to the `Payload Injection' pipeline.


\subsection{E-Mail Form Retriever}
\label{Comp:EMFR}
The E-Mail Form Retriever is the first stage in the Payload Injection Pipeline. It has the following important functions:
\begin{itemize}
	\item Retrieve the newly inserted forms in the `\lstinline{email_forms}' table, checking to ensure no duplication occurs before the fuzzing stage.
	\item Reconstruct each form, using the data stored in the `\lstinline{form}' table, complete with input fields and their values.
	\item Construct the URL for the `action' attribute of the form so that we can send the HTTP request to the correct URL. 
\end{itemize}

\subsection{Fuzzer}
\label{Comp:Fuzzer}
The Fuzzer is the heart of the system and is the only component that interacts directly with the external websites. The Fuzzer is split into smaller modules, each of which is responsible for a particular type of fuzzing.  We inject payloads in two different stages, to improve the efficiency, and reduce the total number of HTTP requests we generate. This is because making HTTP requests is an expensive process \cite{McGrath2009}, and can be a cause of bottlenecks in a Crawler-Fuzzer system \cite{ShkapenyukTorstenSuel2001}.
The two different types of payloads we use for fuzzing are:
\paragraph{Non-Malicious Payload}
\label{Comp:Fuzzer:nmp}
The regular or non-malicious payload is a straight forward E-Mail address of the format -- `reguser(xxxx)@example.com', where `xxxx' is replaced by our internal `\lstinline{form_id}', to create a one-to-one mapping of the payloads to the forms, and `example.com' is replaced by the required domain. In our case, this domain was `wackopicko.com'.
This non-malicious payload allows us to check whether we can inject data into a form and whether we can overcome the `anti-bot' measures on the given website, without attempting to fuzz the website. 

\paragraph{Malicious Payload}
\label{Comp:Fuzzer:mp}
In the malicious payload scenario, we inject the fields with the \dq{\texttt{bcc}} (blind carbon copy) element. If the vulnerability is present, this will cause the server to send a copy of the e-mail to the e-mail address we added as part of the `bcc' field.

We consider a special case: the addition of a \dq{\texttt{x-check:in}} header field to the payloads. This is due to Python's exhibited behavior when attaching
headers. Instead of overwriting a header if it is already present, it ignores duplicate headers. So, in case the \dq{\texttt{bcc}} field is already present as part of the headers, our injected \dq{\texttt{bcc}} header would be ignored. To overcome this, we need to inject a new header that is not likely to be generated by the web application. Hence, we inject our own \dq{\texttt{x-check:in}} header to ensure we can get results if the injection was successful.

The malicious payloads consist of 4 different payloads. Each of these payloads is crafted for a particular use case. The four payloads are:
\lstset{language=html}
\begin{enumerate}
	\item
	%inside the lstinline, we need to use \\ for \textbackslash
	\colorbox{lightgray}{\lstinline{nuser(xxxx)@wackopicko.com\\nbcc:maluser(xxxx)@wackopicko.com}} - This is the most minimal payload, it injects a `newline' character followed by the `bcc' field.
	\item \colorbox{lightgray}{\lstinline{nuser(xxxx)@wackopicko.com\\r\\nbcc:maluser(xxxx)@wackopicko.com}} - This payload is added for purposes of cross-platform fuzzing: `\textbackslash{}r\textbackslash{}n' is the `Carriage Return - New Line (CRLF)' used on Windows systems. 
	\item \colorbox{lightgray}{\lstinline{nuser(xxxx)@wackopicko.com\\nbcc:maluser(xxxx)@wackopicko.com\\nx-check:in}} - As discussed above, the addition of the \dq{\texttt{x-check:in}} header is to inject Python based websites.
	\item \colorbox{lightgray}{\lstinline{nuser(xxxx)@wackopicko.com\\r\\nbcc:maluser(xxxx)@wackopicko.com\\r\\nx-check:in}} - Same as the previous payload, but containing the additional `\textbackslash{}r' for Windows compatibility.
	
\end{enumerate}
The `xxxx' in all of the payloads is replaced by our internal `\lstinline{form_id}', so as to create a one-to-one mapping of the payloads to the forms. The coverage provided by each payload is shown in Table~\ref{tab:payloadcov}.\\

\begin{table}[!htbp]
	\centering
	\begin{tabular}{|c|c|c|}
		\hline
		\multicolumn{1}{|c|}{\textbf{Payload}} & \multicolumn{1}{c}{\textbf{Languages covered}} & \multicolumn{1}{|c|}{\textbf{Platforms covered}}\\
		\hline
		1 & PHP, Java, Ruby, etc. & Unix\\
		\hline
		2 & PHP, Java, Ruby, etc. & Windows\\
		\hline
		3 & Python & Unix\\
		\hline
		4 & Python & Windows\\
		\hline
	\end{tabular}
	\caption[\titlecap{Payload coverage}]{Payload coverage, each payload covers a different platform/language.}
	\label{tab:payloadcov}
\end{table}
Along with the payload, the Fuzzer also injects data into the other fields of the form. This data must pass validation constraints on the individual input fields e.g.,\ for a name field, numbers might not be allowed. It is essential that the data we inject into the input fields adhere to the constraints. Our Fuzzer does this by making use of a `Data Dictionary' which has predefined `keys' and `values' for standard input fields such as \texttt{name}, \texttt{date}, \texttt{username}, \texttt{password}, \texttt{text}, and \texttt{submit}. The default values for these are generated on-the-fly for each form, based on generally followed guidelines for such fields. For example, password fields should consist of at least one uppercase letter, one lowercase letter, and a special character.

Once the data (including the payload) for the form is ready, the Fuzzer constructs the appropriate HTTP request (GET or POST) and sends the HTTP request to the URL that was generated by the E-Mail Form Retriever (Section~\ref{Comp:EMFR}). 


\subsection{E-Mail Analyzer}
\label{Comp:EMA}
The E-Mail Analyzer checks for the presence of injected data in the received e-mails. This module works on the e-mails received and stored by our Postfix server, and depending on the user who received the e-mail, it performs different functions.
\paragraph{Analyzing regular e-mail}
`Regular e-mail' refers to the e-mails received by the \colorbox{lightgray}{\lstinline{reguser(xxxx)@wackopicko.com}} --- where `xxxx' is our internal `\lstinline{form_id}' --- that were sent due to injecting the `regular or non-malicious' payload (discussed in Section~\ref{Comp:Fuzzer:nmp}). The objective of the analysis on this e-mail is identify if the input fields that we injected with data appear on the resulting e-mail, and if so, which fields appear where.

To find this, we read through each received e-mail, and check whether \emph{any} of the fields we injected with data appear as part of either the headers or the body of the e-mail. If they do, we add them to the list of fields that can potentially result in an E-Mail Header Injection for the given e-mail. we then pass on this information back to the Fuzzer pipeline, along with the `\lstinline{form_id}', so that the Fuzzer can now inject the malicious payloads into the same form.

\paragraph{Analyzing e-mail with payloads}
The `e-mails with payloads' refer to e-mails received by either the \colorbox{lightgray}{\lstinline{nuser(xxxx)@wackopicko.com}} or \colorbox{lightgray}{\lstinline{maluser(xxxx)@wackopicko.com}} accounts. These e-mails were received due to injecting the malicious payloads that were discussed in Section. \ref{Comp:Fuzzer:mp}. Analysis of these e-mails is considerably simpler than that of the regular e-mails. This is due to the fact that this involves lesser processing of the contents of the e-mail compared to the previous section.
\subparagraph{Detecting injected bcc headers}
As discussed in the payloads section (\ref{Comp:Fuzzer:mp}), the payloads were crafted in such a way that the e-mails received by `maluser' account directly indicate the presence of the injected `bcc' field. Thus, we simply parse the E-Mails and store them in the Database.

\label{analyze:detect_x_check}
\subparagraph{Detecting injected x-check headers}
E-Mails not received by the `maluser' account but by the `nuser' account constitute a special category of e-mails.
These e-mails could have been generated due to two reasons:
\begin{enumerate}
	\item The websites performed some sanitization routines and stripped out the \dq{\texttt{bcc}} part of the payload, thereby sending e-mails only to the `nuser' account. These e-mails then act as proof that the vulnerability was not found on the given website.
	\item A more conducive scenario is when the \dq{\texttt{bcc}} header was ignored for some reason, e.g.\ Python's default behavior when it encounters duplicate headers. In this case, we check whether the e-mail contains the custom header \dq{\texttt{x-check}}. If it does, then this is a successful exploit of the vulnerability, and we store it in the database.
\end{enumerate}
\subsection{Database}
We collect and store as much data as possible at each stage of the pipeline. This is due to the two following reasons:
\begin{enumerate}
	\item The data is used to validate our findings.
	\item The data collected can be used for other research projects in this area.
\end{enumerate}
Each table in our database is listed in Table~\ref{tab:dbtf} along with the data it is designed to hold. A schema of the database is shown in Figure~\ref{fig:dbschema}.


\begin{table}[!htbp]
	\centering
	\begin{tabular}{|p{1cm}|p{5cm}|p{8cm}|}
	\hline
	\multicolumn{1}{|c|}{\textbf{S.No}} & \multicolumn{1}{c}{\textbf{Table Name}} & 
	\multicolumn{1}{|c|}{\textbf{Purpose}}\\
	\hline
	1 & form & To hold data about all the forms that we receive from the Form Parser.\\
	\hline
	2 & {{email\_forms}} & Holds the output of the E-Mail Field Checker, i.e.,\ references to the ID's of the forms that contain E-Mail fields. \\
	\hline
	3 & params & Holds the actual input fields of the forms, including their default values.\\
	\hline
	4 & {{fuzzed\_forms}} & Holds the data of the forms that were injected, including the payload used to inject and the URL to which the HTTP Request was delivered.\\
	\hline
	5 & {{received\_emails}} & Contains data about the E-Mails received for the regular payload, including which injected data fields were present in the E-Mail.\\
	\hline
	6 & {{successful\_attack\_emails}} & Contains data about the E-Mails received for the malicious payload. This contains the end result of the payload injection pipeline.\\
	\hline
	7 & requests & Contains data about the requests generated for each URL.\\
	\hline
	8 & {{blacklisted\_urls}} & Used for skipping certain websites that may blacklist our Crawler-Fuzzer.\\
	\hline
\end{tabular}        
	\caption[\titlecap{Database - tables}]{The different tables in our database.}
	\label{tab:dbtf}
\end{table}

\begin{figure}[!htbp]
	\centering
	\includegraphics[width=15cm, height=11cm]{System/dbschema}
	\caption[\titlecap{Database schema}]{Database schema.}
	\label{fig:dbschema}
\end{figure}


%\section[Issues]{Design Issues}
\label{sys:issues}
This section will describe the issues we faced with the design decisions we made, and how we did our best to mitigate them, and their effect on the system.

\begin{itemize}
	\item \label{issues:fpr}False Positive rate for the E-Mail Field Checker\\
	As discussed in Section \ref{Comp:EMFC}, we only search for the words `email', `mail' or `e-mail' (case insensitive) inside the forms to detect the presence of e-mail fields, instead of searching for an \colorbox{lightgray}{\lstinline{<input type = email>}}. This might result in a false positive in certain forms, like the one shown in Listing. \ref{code:false_positive}.
	
	\lstset{language=HTML,caption={E-Mail field checker - false positives, the system\\incorrectly classifies this as an e-mail form.},label={code:false_positive}}
	\begin{lstlisting}
	<form method="post">
	E-Mail us if you have any questions!!
	<input type="text" name="query"><br>
	<input type="submit" value="Search">
	</form>
	\end{lstlisting}
	
	The word `E-Mail' on Line 2 will result in our system classifying this form as a potential e-mail form, while it clearly is not. However, as we will see, this is not really a significant issue, as despite being added to the \lstinline{`email_forms'} table, this form will never be injected in the `fuzzer' due to the absence of the appropriate input field in the form. We chose to go with this design, as it allows us to detect almost every form that provides the capability to send or receive e-mail.
	
	\item Parallelism for the system\\
	\label{issues:parallel}
	Every component in the pipeline benefits hugely from parallel processing of the data. However, Python's GIL (Global Interpreter Lock) does not allow the running of multiple native threads concurrently. To overcome this, we used a Celery task queue (discussed in Section \ref{exp:Celery}), which allowed a level of parallelism that Python does not provide by default. Even though this makes the system faster than a single-threaded approach, it still leaves room for improvement in terms of performance. Despite the speed drop that results from lack of full parallelism, we chose to go with Python, for the raw power it provides, its text processing capabilities, PCRE (Perl Compatible Regular Expressions) compatibility, and the numerous libraries available for parsing HTML, interfacing with databases and generating HTTP requests.
	
	\item URL Construction\\
	The multiple ways in which a URL is specified (i.e.\ Relative and Absolute URLs) complicates the construction of the URL from the `action' attribute of the form.  As an example, the following URLs are all equivalent (as parsed by a browser, assuming we are in the path `www.website.com'):
	
	\begin{itemize}
		\item \colorbox{lightgray}{\lstinline{action=mail.php}}
		\item \colorbox{lightgray}{\lstinline{action=./mail.php}}
		\item \colorbox{lightgray}{\lstinline{action=http://website.com/mail.php}}
		\item \colorbox{lightgray}{\lstinline{action=www.website.com/mail.php}}
	\end{itemize}
	Add to this, if the form is a self-referencing form\,\footnotemark, and is present in mail.php, the following are equivalent to the above URLs as well:
	\footnotetext{A self-referencing form is one which submits the form data to itself. It includes logic to both display the form and process it. It is a \emph{very} common feature in PHP-based scripts.}
	\begin{itemize}
		\item \colorbox{lightgray}{\lstinline{action=""}}
		\item \colorbox{lightgray}{\lstinline{action=#}}
		\item `action' is completely omitted
	\end{itemize}
	Also, relative URLs pose another problem. If the URL of the form page ends with `/' and the `action' specifies a path starting with `/' (illustrated in Listing \ref{issues:url}), we would need to strip one of the two slashes. This increases the overall complexity of our URL generator, as we have to account for all these possibilities.
		
	\lstset{language=HTML,caption={URL construction, the resulting url needs to be www.website.com/mail.php and not www.website.com//mail.php},label={issues:url}}
	\begin{lstlisting}
	Current URL = www.website.com/
	<form action=/mail.php>
	\end{lstlisting}
	
	As using a browser engine to reconstruct these URLs  and connecting it to the fuzzer pipeline would have added unnecessary bulk to the project, we chose to go with a best-effort approach to this problem, where our system covers all these possibilities with a lightweight URL Generator, however, we cannot know for certain whether this works for other unforeseen ways of specifying a URL.
	
	\item Black-box Testing\\
	The approach that we have selected --- Black-box testing --- is highly beneficial as explained in Section \ref{sys:appr}. However, it also has a drawback in that we cannot verify whether the reported vulnerability exists in the source code or is a feature of the website (e.g., the website allows users to send bulk e-mail, adding as many \dq{\texttt{cc}} or \dq{\texttt{bcc}} headers). We have to manually e-mail the developers to get this feedback.
	
	\item Mapping responses to requests\\
	As we are generating multiple payloads for each form, and the received e-mail may not contain the name of the domain from which we received the e-mail, it is difficult to map the response e-mails to the right requests. We instead use the `\lstinline{form_id}' as part of the payload to map responses to requests accurately.
	
	\item Bot Blockers\\
	\label{issues:captcha}
    Because our system is fully automated, it is also susceptible to being stopped by `bot-blockers' i.e.\ mechanisms built-in to a website to prevent automated crawls or form submissions. Measures like CAPTCHA (Completely Automated Public Turing test to tell Computers and Humans Apart) and hidden form fields are often used to detect bots \cite{captchas3}, \cite{captchas2}.
    
    We have made sure that we do not affect hidden fields in the form, however, we do not have an anti-CAPTCHA functionality built into our system, and thus our system will not test such websites.
    
	\item Handling Malformed HTML\\
    The parser that we use for HTML parsing --- Beautiful Soup --- does not try to parse malformed HTML, and throws an exception on encountering malformed content. Thus, we have designed the system to exit gracefully on such occasions. A side-effect of this is that our system is unable to parse websites with bad markup\,\footnotemark.
    
    \footnotetext{We do not have any data about whether bad markup indicates an overall lower quality of the website, and thus cannot comment on whether such websites are more likely to have vulnerabilities, although the author strongly suspects that that might be the case.}
	
	\item Crawling WordPress and other CMS-based websites\\
	\label{issues:cms}
	In contrast to bot blockers that try to prevent the automated systems from attacking them, WordPress and other CMS based websites use a blacklisting approach to prevent bot attacks. Unfortunately, because we generate multiple requests to each website, this results in our IPs getting blacklisted. To overcome this, we did two things:
	\begin{enumerate}
		\item Used an IP range of 60 different IP addresses. 
		\item Used a blacklist of our own to prevent our Fuzzer from fuzzing websites that are known to blacklist automated crawlers.
	\end{enumerate}
\end{itemize}
%\section{Assumptions}
We made certain assumptions while building the system. This section describes the assumptions and explores to what extent these hold true:
\begin{enumerate}
	\item \textbf{Crawler is not blocked by firewalls}\\
	This is a requisite for our system to work. If the Crawler is blocked for any reason, we do not get the data feed for our system, and without this input, it is almost impossible to set our system up.
	\item \textbf{The Crawler feed is an ideal representation of the World Wide Web} \\
	This is a reasonable expectation, albeit an unrealistic one.
	
	It is unrealistic because Crawlers work on the concept of proximity. They detect for the presence of In-Links and Out-Links from a particular URL, and hence the returned URLs are usually related to each other (at least the ones that are returned adjacent to each other).
	
	However, this assumption is reasonable due to the `Law of averages' \cite{wiki:Law_of_averages}, the `Law of big numbers' \cite{wiki:Law_of_large_numbers}, and the concept of `Regression to the mean' \cite{wiki:Regression_toward_the_mean}. Simply stated, a crawl of this large magnitude should give us a very distributed sample of the overall Web, eventually converging to the average of all websites in existence.
	
	\item \textbf{Injection of \dq{\texttt{bcc}} indicates the existence of E-Mail Header Injection Vulnerability} \\
	We assume that the ability to inject a \dq{\texttt{bcc}} header field is proof that the E-Mail Header Injection vulnerability exists in the application. We do not inject any additional payloads that can modify the subject, message body, etc.\ as this analysis is designed to be as benign as possible.
	We believe that this is a reasonable assumption, as altering e-mail headers is a goal of exploiting E-Mail Header Injection vulnerability.
\end{enumerate}

That concludes our discussion about the design of the system. To recap, we discussed our approach, the system architecture and how the components fit into our architecture. We also discussed the issues faced, and the assumptions that we made while building the system. %The next chapter describes, in brief, the experimental setup we used for our system.


%	\chapter{Evaluation}
This chapter describes the experimental setup for our project including the servers used, the software and the platforms involved, the languages used, and the task queue system that was used for parallelism. We follow this up with our evaluation of the system, with a test suite, and proof of concept examples.

Our system was run on a server with the following configuration: Dell
PowerEdge T110 II Server, CPU: Intel(R) Xeon(R) CPU E3-1220 V2 @
3.10GHz, Cache Size: 8,192 KB, No. of Cores : 4, Total Memory (RAM) : 16 GB.

\section[Platform]{Platforms and Software}

We enumerate the platforms and the software used for our project in Table~\ref{tab:platsw}.
\begin{table}[!htbp]
	\centering
	\begin{tabular}{|c|c|}
		\hline
		Operating system & Ubuntu 14.04\\
		\hline
		Server & Apache - 2.4.17\\
		\hline
		Database & MariaDB - 10.1.9\\
		\hline
		Mail Server & Postfix - 2.11.0\\
		\hline
		Other software used & Mailcatcher, PostMan, HTTPRequester, RabbitMQ\\
		\hline
	\end{tabular}
	\caption[\titlecap{Platforms and software}]{Platforms and software used for our project.}
	\label{tab:platsw}
\end{table}
\section{Languages Used}

We used Python 2 to build the system. The following factors influenced our choice of language: text processing capabilities, PCRE (Perl Compatible Regular Expressions) compatibility, and the numerous libraries for HTML Parsing, HTTP request generation, mail processing etc.
We made use of the following major libraries (shown in Table~\ref{tab:libs}) for our system.

\begin{table}[!htbp]
	\centering
	\begin{tabular}{|c|c|}
		\hline
		\multicolumn{1}{|c|}{\textbf{Library}} &
		\multicolumn{1}{c|}{\textbf{Functionality}} \\
		\hline
		Requests & HTTP Request Generation\\
		\hline
		Beautiful Soup & HTML Parsing\\
		\hline
		Mailbox & Mail Processing\\
		\hline
		Celery & Task Queues\\
		\hline
	\end{tabular}
	\caption[\titlecap{Python libraries}]{Libraries that we used and their functions.}
	\label{tab:libs}
\end{table}

Despite the many benefits that Python 2 provides, we had certain issues with the language --- discussed in Section \ref{issues:parallel} --- such as Python's GIL (Global Interpreter Lock) which does not allow the running of multiple native threads concurrently.
The following section (Section \ref{exp:Celery}) describes in detail the task queue system (Celery) that we used to overcome this limitation of Python.

\section{Celery Queues}
\label{exp:Celery}
We used a Celery task queue running on RabbitMQ to overcome the GIL. According to Celery Project Homepage \cite{Celery}:
\begin{quotation}
``Celery is an asynchronous task queue/job queue based on distributed message passing.''
\end{quotation}

Simply put, Celery allows us to process multiple tasks in parallel by making use of what is known as a task queue. Celery instantiates multiple workers that listen to these queues and processes each task individually. This simulates pseudo-parallel processing to a certain degree, by allowing us to run multiple instances of the same program. It does this by using a message broker called RabbitMQ.
According to RabbitMQ's Wikipedia page \cite{wiki:RabbitMQ},
\begin{quotation}
``RabbitMQ is an open source message broker software that implements the Advanced Message Queuing Protocol (AMQP)''
\end{quotation}
RabbitMQ facilitates the storage and transport of messages on queuing systems. It is also cross-platform and open source, providing us with clients and servers for many different languages, thereby being the ideal fit for Celery.
Thus, by using Celery and RabbitMQ together, we were able to achieve a certain degree of parallelism that would not have been possible with traditional Python.

%The next chapter presents our results and showcases our analysis of the said results.

\section{Test Suite}
\label{Arch:Test}
The test plan for our system includes a set of unit tests for each module in the pipeline. Further, we have unit tests for every %every instead of each
individual function in the modules. The functions are tested separately, using mocks and stubs, so as to ensure isolated testing.
This section outlines the test plan in the following manner. We list the modules that are tested, and then describe what each unit test tests for.
\begin{itemize}
	\item Form Parser
	\begin{itemize}
		\item \colorbox{lightgray}{\lstinline{test_url_exception}} - Tests whether the system handles incorrect or malformed URLs properly and terminates cleanly.
		\item \colorbox{lightgray}{\lstinline{test_db_connection}} - Tests whether the Database Connection is set up and queries can be executed.
		\item \colorbox{lightgray}{\lstinline{test_form_parser}} - Tests for the proper parsing of HTML, and if the system exits cleanly in case parsing is not possible.
	\end{itemize}
	
	\item E-Mail Field Checker
	\begin{itemize}
		\item \colorbox{lightgray}{\lstinline{test_check_for_email}} - Tests whether the system finds E-Mail fields in the form when the words `e-mail' or `email' are present in the form (case insensitive).
		\item \colorbox{lightgray}{\lstinline{test_check_for_no_email}} - Tests whether the system finds no E-Mail fields when the words `e-mail' or `email' are \emph{not} present in the form (case insensitive).
	\end{itemize}
	
	\item E-Mail Form Retriever
	\begin{itemize}
		\item \colorbox{lightgray}{\lstinline{test_reconstruct_form}} - Tests for the proper reconstruction of the form stored in the Database.
		\item \colorbox{lightgray}{\lstinline{test_construct_url}} - Tests whether the URL for submission was constructed properly, includes checks for relative URLs, absolute URLs, and presence of \lstinline{`base'} tags.
		\item \colorbox{lightgray}{\lstinline{test_email_form_retriever_already_fuzzed}} - Tests for duplicate fuzz requests, and whether the system rejects these requests.
		\item \colorbox{lightgray}{\lstinline{test_email_form_retriever_calls_fuzzer_for_new_fuzz}} - Tests whether the E-Mail Form Retriever calls the Fuzzer module with the proper data when it gets a new fuzz request.
	\end{itemize}
	
	\item Fuzzer
	\begin{itemize}
		\item \colorbox{lightgray}{\lstinline{test_send_get_request}} - Tests for the proper handling of GET requests.
		\item \colorbox{lightgray}{\lstinline{test_send_post_request}} - Tests for the proper handling of POST requests.
		\item \colorbox{lightgray}{\lstinline{test_correct_fuzzer_data}} - Tests whether the payload generated for the given form data is correct and consistent. Also tests whether the payload was part of the resulting HTTP request.
		\item \colorbox{lightgray}{\lstinline{test_incorrect_fuzzer_data}} - Tests for incorrect form data, and ensures that a payload does not end up in the wrong input field in the resulting HTTP request.
	\end{itemize}
	\item E-Mail Analyzer
	\begin{itemize}
		\item \colorbox{lightgray}{\lstinline{test_load_mail}} - Tests whether the E-Mails are loaded and parsed correctly by the E-Mail Analyzer.
		\item \colorbox{lightgray}{\lstinline{test_parse_headers}} - Tests for the proper parsing of headers present in the E-Mail.
		\item \colorbox{lightgray}{\lstinline{test_analyze_regular_mail}} - Tests whether the E-Mail Analyzer parses the regular E-Mail properly and extracts the injected input fields that are present in the E-Mail.
		\item \colorbox{lightgray}{\lstinline{test_analyze_malicious_mail}} - Tests whether the E-Mail Analyzer parses the E-Mails received due to the malicious payloads properly, is able to extract the `bcc' headers, and is able to link them to the proper fuzzing request and payload.
		\item \colorbox{lightgray}{\lstinline{test_analyze_x_check_header}} - Tests whether the `x-check' header is read by the E-Mail Analyzer.
	\end{itemize}
\end{itemize}
The unit tests were written using Python's built-in `Unittest' module, mocking was done using the built-in `MagicMock' module. The tests allow us to be reasonably certain that our system works as expected.

To validate our system, we constructed three sets of web applications in PHP, Python, and Ruby. Each of these applications was a simple web application that accepted user input to construct and send an \Email.

The server-side code for PHP is shown in Listings~\ref{code:phpemi},
while the Python and Ruby code is not shown for space limitations.

% Adam: I think it might be nice to have the Ruby code here too. - DONE.

Before performing a wide scan of the web, we verified that our system was able to detect the \ehi vulnerabilities present in all the sample web applications. 

%% We tested for the headers being injected in real-time by running an instance of MailCatcher, set to listen on all SMTP messages. A sample screenshot of a fuzzed request for the Ruby backend (generated in PostMan) is shown in Figure~\ref{fig:postmanruby}. The \email sent due to injecting this payload (as captured by MailCatcher) is shown in Figure~\ref{fig:mailcatcherruby}. It can be seen that the headers have been added to the resulting \email, and we have successfully managed to overwrite the \texttt{Subject} field with our message, `hello'.

%% The astute reader might have noticed that in the given example we have used \texttt{\%0a} to separate the headers, while in Section~\ref{Comp:Fuzzer}, we had used \texttt{\textbackslash{}n}. This is due to URL encoding~\cite{rfc1738}, wherein special characters in th URL are `encoded' or `escaped' with their ASCII equivalent.
%% The reason why we do not have to do this with the payloads our system injects is due to the fact that the Python Requests library that we use to generate the HTTP requests automatically does this encoding for us.

%% \begin{lstlisting}[language=HTML,caption={HTML page for showcasing
%%       \ehi, a simple front-end for our
%%       examples.},label={code:html}, float]
%% <!doctype html>
%% <html lang="en">
%% <head>
%% <meta charset="utf-8">
%% <meta name="author" content="XYZ">
%% <title>Mock Email</title>
%% </head>
%% <body>
%% <form action="script-path" method="post">
%% <input type="text" name="email">
%% <textarea name="msg"></textarea>
%% <input type="submit" value="Email Me!">
%% </form></body></html>
%% \end{lstlisting}
%\lstset{language=Ruby,caption={Ruby program with e-mail header injection vulnerability.},label={code:rubyemi}}
\begin{lstlisting}
require 'sinatra'
require 'net/smtp'

get '/hello' do
email = params[:email]

message = """
From: Sai <schand31@asu.edu>
Subject: SMTP e-mail test
To: #{email}

This is a test e-mail message.
"""
# construct a post request with email set to attack_string
# attack_string => sai@sai.com%0abcc:spc@spc.com%0aSubject:Hello
Net::SMTP.start('localhost', 1025) do |smtp|
smtp.send_message message, 'schand31@asu.edu',
'to@todomain.com'
end
# Headers get added, and Subject field changes to what we set.
end
\end{lstlisting}

%% \begin{figure}[tbp]
%% 	\centering
%% 	\includegraphics[width=\linewidth]{System/EMI_Postman_Ruby}
%% 	\caption[\titlecap{Fuzzing a request for the Ruby backend}]{Fuzzing a request for the Ruby backend, the payload can be seen inside the address bar.}
%% 	\label{fig:postmanruby}
%% \end{figure}

%% \begin{figure}[tbp]
%% 	\centering
%% 	\includegraphics[width=\linewidth]{System/EMI_Mailcatcher_Ruby}
%% 	\caption[\titlecap{\Email Header Injection proof of concept - Ruby}]{\Email header injection proof of concept - Ruby, we can see that multiple headers (bcc, x-check, subject) have been inserted into the resulting \email.}
%% 	\label{fig:mailcatcherruby}
%% \end{figure}

	
	\chapter[Results]{Data Analysis and Results}
This chapter serves to present our findings: the data that we gathered from our crawl, the data generated due to the fuzzing attempts and our analysis on this data.
\subsection{Collected Data}

% TODO: Need to put in a number here, and discuss the high level
% crawl. 
We ran the system for XYZ days/hours/whatever 

From our extensive crawl of the web, we were able to gather the data
shown in Table~\ref{tab:data}. Our system crawled \urls unique URLs,
found a total of \forms\ forms from \uniqueforms\ unique domains, and
found \emailforms\ forms that contained an \email field from \uniqueemailforms\ unique domains.

\begin{table}[tbp]
	\centering
	\scriptsize
	\begin{tabular}{|c|c|}
		\hline
		\multicolumn{1}{|c|}{\textbf{Type of Data}} &
		\multicolumn{1}{c|}{\textbf{Quantity}}\\
		\hline
		URLs Crawled & \urls \\
		\hline
		Total Forms found & \forms \\
		\hline
		Forms with E-Mail Fields & \emailforms \\
		\hline
	\end{tabular}
	\caption[\titlecap{Collected data}]{The data collected for our
      project.}
    \vspace{-5ex}
	\label{tab:data}
\end{table}



\subsection{Fuzzed Data and Received \Emails}
Table~\ref{tab:fuzzed_data} shows the quantity of \emails that we received for the benign and malicious payloads. 
\begin{table}[tbp]
	\centering
	\scriptsize
	\begin{tabular}{|c|c|c|}
		\hline
		\multicolumn{1}{|c|}{\textbf{Type of fuzzing}} &
		\multicolumn{1}{c|}{\textbf{Forms fuzzed}} &
		\multicolumn{1}{c|}{\textbf{E-Mails received}}\\
		\hline
		Regular payload & \fuzzed & \recd \\
		\hline
		Malicious payload & \malfuzzed & \success \\
		\hline
	\end{tabular}
	\caption[\titlecap{Fuzzed data}]{The data we fuzzed and the e-mails we received.}
    \vspace{-5ex}    
	\label{tab:fuzzed_data}
\end{table}

\paragraph{\Email received from forms}
The \emails that we received can be categorized into two categories:
\begin{enumerate}
	\item \Emails due to regular payload\\
	This represents the total number of web applications that sent \emails to us. This indicates that we were able to successfully submit the forms on these sites to trigger the web application to send an \email.
	
	\item \Emails due to malicious payload\\
    Once we receive an \email from a web application due to the regular payload, we fuzz those forms with the malicious payloads. This field represents the total number of unique URLs that are  contain an \ehi vulnerability.
\end{enumerate}




\subsection{Analysis of the received e-mail data}
During our analysis of the received e-mails, we found that the e-mails that we received belonged to one of three broad categories:
\begin{enumerate}
	\item E-Mails with the \texttt{bcc} header successfully injected\\
	This form of injection was our initial objective and we found 265 such e-mails in our received e-mails. This indicates that the websites that sent out these e-mails are vulnerable to e-mail header injection, where we could inject and manipulate any header.
	
	\item E-Mails with the \texttt{to} header successfully injected\\
	We discovered an unintended vulnerability which we would like to christen \texttt{To~header injection}. These injections reflect the ability to inject any number of e-mail addresses into the \texttt{to} field while being unable to inject any other header into the e-mails. We attribute this behavior to inconsistent sanitization by the application. 
	The vulnerability is further aided by the leniency shown by mail servers, wherein they parsed malformed e-mail addresses and delivered it to the right mail server, and on the receiving end, the mail was delivered to the right mailbox. 
	
	While not allowing us complete control over the e-mails sent, \texttt{To header injection} makes it possible to append any number of e-mail addresses, thereby enabling us to leak information, and/or perform DoS (Denial of Service) attacks.
	
	\item E-Mails with the \texttt{x-check} header successfully injected\\
    The third category of e-mails received were e-mails with the \texttt{x-check} header injected. As discussed in Section~\ref{analyze:detect_x_check}, 
    these let us differentiate between unsuccessful attempts and successful attempts by injecting the additional header, allowing us to check whether headers other than the \texttt{bcc} header can be injected into the generated e-mail. 
\end{enumerate}
We list each category and the number of e-mails received by the category in Table~\ref{tab:analysis}. 

\begin{table}[!htbp]
	\centering
	\begin{tabular}{|c|l|c|}
		\hline
		\multicolumn{1}{|c|}{\textbf{S.No}} &
		\multicolumn{1}{c|}{\textbf{Type of Injection}} &
		\multicolumn{1}{p{3cm}|}{\centering \textbf{No. of e-mails received}}\\
		\hline
		1 & E-Mail Header Injections with `bcc' header & 265\\
		\hline
		2 & E-Mail Header Injections with `x-check' header & 214\\
		\hline
		3 & `To header' injections alone & 142\\
		\hline
		4 & E-Mail Header Injections with `bcc' and `x-check' headers & 188\\
		\hline
		5 & Both `To header' injections and x-check headers &
		11\\
		\hline
		6 & `x-check' headers found in `nuser' e-mails & 56\\
		\hline
		7 & Unique `x-check' headers found in `nuser' e-mails & 26\\
		\hline
		8 & Total successful injections (1 + 3 + 7) & 433\\
		
		\hline
	\end{tabular}
	\caption[\titlecap{Analysis of the data}]{Classification of the e-mails that we received into broad categories of the vulnerability.}
	\label{tab:analysis}
\end{table}
We explain the combination of these header injections (4-7) as follows:
\begin{itemize}
	\item E-Mail Header Injections with \texttt{bcc} and \texttt{x-check} headers\\
	These represent the perfect attack scenario where we are able to inject multiple headers into the e-mails. We can see that over 70\% of the received \texttt{bcc} header injected e-mails are also susceptible to other header injections.
	
	\item Both \texttt{To} header injections and \texttt{x-check} headers \\
	This combination shows us that in addition to being able to inject into the \texttt{To} fields, we are able to inject additional headers into the e-mail. It is not clear what causes this behavior; however, these can be exploited to achieve the same result as a regular E-Mail Header Injection.
	
	\item \texttt{x-check} headers found in \texttt{nuser} e-mails\\
	In addition to analyzing the \texttt{maluser} account, we also analyze emails received by the \texttt{nuser} account. We explain the presence of these headers in the following paragraph.

	\item Unique \texttt{x-check} headers found in \texttt{nuser} e-mails\\
	These represent the e-mails with \lstinline|form_ids| that were \emph{not} already found in the \texttt{maluser} account. We attribute these e-mails to (probably) having a backend that was built with Python or another language having a similar behavior with respect to constructing headers.
	
	\item Total successful injections\\
	This represents the total number of successful injections our system made. This includes the E-Mail Header Injections with \texttt{bcc} header\,(1), \texttt{To} header injections alone\,(3), and Unique \texttt{x-check} headers found in \texttt{nuser} e-mails\,(7). This is the total number of vulnerabilities that were found by our system.
	
\end{itemize}

\subsection[The Pipeline]{Understanding the Data Pipeline}
%This section serves to represent our pipeline quantitatively and graphically. 
Table~\ref{tab:pipeline} showcases the data gathered by our pipeline, with the differential changes at each stage of the pipeline. At each stage of the pipeline, the amount of data decreases, for instance, out of the \urls\ URLs we crawled, only \forms\ forms (\formsDelta) were found. Out of these, only \emailforms\ forms (\emailformsDelta) contained e-mail fields.

In our fuzzing attempts, the same behavior is observed. We fuzzed \fuzzed\ forms with the regular payload, which resulted in a total of \recd\ e-mails~(\recdDelta). After analysis of the received e-mails, we further fuzzed \malfuzzed\ forms, which resulted in \success\ e-mails (\successDelta) which contain the vulnerability across \ips IP addresses from \domains domains.

We attribute the difference in the number of forms found to the number of forms fuzzed (a difference of \diffFoundFuzz forms) to the presence of bot-blocking mechanisms on a website (discussed in Section~\ref{limitations}), though we do not know what percentage was caused by the individual bot-blocking mechanisms discussed in Section~\ref{limitations}. 

We would like to remark that over 1\% of the forms that were not fuzzed (100 out of \diffFoundFuzz) were also tested manually using PostMan to generate HTTP requests with payloads to verify that our system was working as intended.

\begin{table}[tbp]
	\centering
	\scriptsize
	\begin{tabular}{|l|c|c|}
		\hline
		\textbf{Pipeline Stage} & \textbf{Quantity} & \textbf{Differential}\\
		\hline
		Crawled URLs  & \urls & $\Delta$ d2/d1 * 100 \\
		\hline
		Forms found  & \forms & \formsDelta \\
		\hline
		E-Mail Forms found  & \emailforms & \emailformsDelta \\
		\hline
		Fuzzed with regular payload  & \fuzzed & \fuzzedDelta \\
		\hline
		Received e-mails  & \recd & \recdDelta \\
		\hline
		Fuzzed with malicious payload  & \malfuzzed & \malfuzzedDelta \\
		\hline
		Successful attacks  & \success & \successDelta \\
		\hline

	\end{tabular}
	\caption[\titlecap{Data gathered by our pipeline}]{Data gathered
      by our pipeline at each stage}
    \vspace{-5ex}
	\label{tab:pipeline}
\end{table}



% Adam: this is an important part of our contribution, but I don't think that it belongs here. 
%% From our research, it is clear that E-Mail Header Injection is quite widespread as a vulnerability, appearing on \successDelta\ of forms that we were able to perform automated attacks on. This value acts as a lower bound for E-Mail Header Injection vulnerability, and can quite easily be much more if the attacks were of a more concentrated nature, crafted for the individual websites and less automated.




	
	\section{Discussion}
    In the previous section, we discussed our results and presented our analysis of the data gathered. In this section, we discuss the things we learned, and the limitations of our system. We conclude this section with a few ways to mitigate this vulnerability.
\subsection{Lessons Learned}
From our results, it is evident that \ehi vulnerabilities exist in the wild.
% Adam: Sai, I don't understand this math. The 1.46% number is calculated based on the number of forms we were able to fuzz with a malicious payload. This isn't the same as the number of websites. But, using 673/21675680 is not good either because that is # of URLs and it seems like the number you have here is websites. What we need is what % of *websites* (probably estimated as domains) did we find to be vulnerable out of *all* websites/domains that we found. When we use this number, we need to be clear what it is that we are using. -- FIxed this to be clear.
Despite its relatively low occurrence rate compared to the more popular SQL Injection and XSS (Cross-Site Scripting), when we consider total number of domains on the World Wide Web--- 1,018,863,952 according to Internet Live Stats~\cite{InternetLiveStats2016} as of early 2016---and calculate \successWebsitesDelta percent (the occurrence rate of \ehi vulnerability calculated from vulnerable domains as found by our system to total number of domains crawled) of that number, this yields 295,693 domains. Of course, extrapolation in this way is not an accurate measure of the prevalence of \ehi vulnerabilities. However, even with as few as a thousand domains affected by this vulnerability, it can still have a disastrous impact on these domains, and also on overall World Wide Web due to the traffic caused by the sheer number of generated e-mails. 
    
%% We believe that one of the reasons for the small percentage of occurrence (compared to SQL Injection or Cross-Site Scripting), can be attributed to what we like to call the `car parking analogy'.
%%     The car parking analogy is something like this: Imagine that we are parking a car on a road that is prone to attacks by thieves. Now, if all the cars were unlocked, the car that is most likely to get stolen is quite unsurprisingly the most expensive one or the one that is easiest to get away with.
    
%%     Now imagine the same thing on the World Wide Web: we have websites that can each have multiple vulnerabilities. Now, it makes sense for an attacker to try and attack websites with more widespread vulnerabilities such as SQL Injection or XSS, rather than attempt to exploit E-Mail Header Injection, seeing as this requires a more concentrated effort, with carefully crafted payloads and a waiting time for the e-mail to be delivered. SQL Injection attacks and XSS attacks are also better documented, with well-known attack vectors, and automated tools to help detect the presence of these vulnerabilities on websites.
    
%%     This also gives more incentive for the website developer to add protection against attacks such as SQL injection and XSS. The developer might then (possibly with the help of a sanitization library) sanitize the user input and remove \emph{all} special characters, including the newline characters (\textbackslash{}n, \textbackslash{}r), which adversely affects E-Mail Header Injection attacks.

%% 	We come to this conclusion because of our discovery of the \texttt{To header injection}. Clearly, this is possible due to incomplete sanitization performed by the application. We suspect that this incomplete sanitization is actually sanitization that is performed for some other vulnerability, and not specifically for E-Mail Header Injection attacks. We would also like to remark that \texttt{To header injection} is not complete E-Mail Header Injection, but only a special subset.
	
%%     Thus, indirectly, this kind of protection against other attacks affects the attempts to perform E-Mail Header Injection. However, this does not completely negate the attempts if the checks are only on the client-side. Also, even with server-side validation, often, the only input fields that are validated are ones that are either inserted into the database (SQL Injection) and the ones that are displayed to the user as part of the web site (XSS).

% Adam: Please add a citation to a CAPTCHA paper - DONE.
	
	%% This does not mean that the vulnerability is not a large threat. In fact, this vulnerability can also have some major consequences, the least of which can be spamming and phishing attacks.
	%% In today's digital world, identity theft has become all the more common. E-Mail Header Injection provides attackers with the ability to easily extract information about users, not just from a server, but from the user himself, by sending him fake messages that look extremely authentic, since these messages are sent by the mail server of the website itself.
    
    We found two different forms of \ehi: the first one is the traditional one, injecting any header into the \email that allows the attacker complete control over the contents of the \email. 
The second attack has not yet been documented and provides the ability to inject multiple \email addresses into the \texttt{To} field. We call this a \texttt{To header injection}. In this  vulnerability, an attacker can add addresses to the \texttt{To} field of the email with newlines separating the \email addresses. We could not determine if this vulnerability is due to unique flaws in each web application or if this vulnerability is due to an implementation issue with a particular language or framework. However, from our preliminary analysis, it is evident that the vulnerable web applications do not share much in common. 

\texttt{To header injection} allows an attacker to extract information that should be private,
% Adam: It's not clear what this means that we have enough data to spoof the few lines of the message. I thought to header injection just controls the TO field, not the message contents. - Fixed, my bad. DONE.
and in some of these cases, able to inject enough data to spoof other headers of the \email message. From Table~\ref{tab:analysis}, information leakage using \texttt{To header injection} was possible on \ehito forms, while spoofing using \texttt{To header injection} was possible on \ehitoxcheck forms.
    
    %% While not being as impactful as the primary vulnerability, this form of the vulnerability does still provide the ability to send \emails to multiple recipients, and can easily result in information leakage or spam generation on a large scale.
    

\subsection[Limitations]{Limitations}
\label{limitations}
		Because our system is fully automated, it is also susceptible to being stopped by mechanisms in web applications that prevent automated crawls or form submissions. A common reason for our fuzzing attempts to fail is the bot-blocking mechanisms built into the web applications. CAPTCHAs (Completely Automated Public Turing test to tell Computers and Humans Apart)~\cite{captchas2} pose a very difficult problem for our system to exploit \ehi, even if it is present.
		Other measures such as hidden form fields and CSRF (Cross-site Request Forgery~\cite{csrf}) tokens are also often used to detect bots~\cite{captchas3, captchas2}.

		We made sure that we do not fuzz hidden fields in the form, and because our system does not depend on authenticated sessions, CSRF tokens do not pose an issue. However, despite considerable active research in breaking CAPTCHAs~\cite{captchas2, captchas}, breaking CAPTCHAs remains out of the scope of this project. 
		
	   Due to the growing emphasis on responsive web applications, more and more web applications are being built with only client-side JavaScript. Even conventional web applications use JavaScript to dynamically insert content and update the pages. This trend means that these dynamically injected HTML components are not part of the initial HTML that is sent to the client by the server.

		Thus, our system will not see dynamically injected forms and hence is unable to detect whether \ehi vulnerabilities are present in these forms. The workaround would be to use a JavaScript engine to query for the \texttt{document.getElementsByTagName('html')[0].innerHTML} (from inside web browser automation tools such as Selenium), then use that as the source HTML. 
		
		%TODO Adam: added a table here, and explanation about performance tradeoffs with Selenium and bs4.
		A comparison of the running times between the different approaches is shown in Table~\ref{tab:perf}. We chose not to use Selenium as it results in a slowdown of \slowSelenium. 
		
		\begin{table}
			\centering
			\scriptsize
			\begin{tabular}{|p{4cm}|c|c|}
				\hline
				\multicolumn{1}{|c}{\textbf{Method}} &
				\multicolumn{1}{|c|}{\textbf{Running time}} &
				\multicolumn{1}{|c|}{\textbf{Slowdown}}
				\\
				\hline
				\centering Using our pipeline & 629.043 & - \\
				\hline
				\centering Our pipeline with Selenium & 919.372 & \slowSelenium \\
				\hline				
				\centering Parsing \email fields instead of `grep'ing& 707.154 & \slowParse \\								
				\hline
			\end{tabular}
			\caption[\titlecap{}]{Running times in seconds for crawling, parsing, and detecting presence of \email fields in 1000 random wikipedia pages.}
            \vspace{-5ex}
			\label{tab:perf}
		\end{table}

        Because we search for the words \texttt{e-mail}, \texttt{mail}, or \texttt{email} within the HTML form, if the website does not use English names for its forms, our system will not be able to find the presence of an \email field. An example is by using the French word for \texttt{e-mail}---courrier {{\'e}}lectronique--- our system is unable to find the presence of the e-mail field. 
        
		During the crawl, our system was blacklisted by a few web
        applications (mostly WordPress ones), and Internet Service
        Providers (ISPs). To overcome this, we did two things: (1)
        used an IP range of 60 different IP addresses, and (2) Used a
        blacklist of our own to prevent our Fuzzer from fuzzing
        applications that are known to blacklist automated crawlers.
        This restricted us from gathering data about these
        applications.


		We found that certain WordPress plugins prevent the \ehi attack by sanitizing user input on contact forms. Although not all  WordPress web applications are secure, between the presence of the plugins on some websites, and getting tagged as ``spambots'' by others, we found few vulnerabilities on WordPress web applications.

        E-Mail libraries such as the PHP Extension and Application Repository's (PEAR) mail library provide sanitization for user input. While this is not strictly a limitation of our project, it still means that we are not able to inject sites that used these libraries.

        The parser that we use for HTML parsing---Beautiful Soup---does not parse heavily malformed HTML and throws an exception on encountering such HTML. Thus, we have designed the system to exit gracefully on such occasions. A side-effect of this is that our system is unable to test web applications with very bad HTML markup.

        Black-box testing is highly beneficial as explained in Section~\ref{sys:appr}, however it also has a drawback in that we cannot verify whether the reported vulnerability exists in the source code or is a feature of the website (e.g., the website allows users to send bulk e-mail, adding many \texttt{cc} or \texttt{bcc} headers). We must manually notify the developers to get this feedback.

%%        	As discussed in Section~\ref{Comp:EMFC}, we only search for the words \texttt{email}, \texttt{mail} or \texttt{e-mail} (case insensitive) inside the forms to detect the presence of e-mail fields, instead of searching for an \colorbox{lightgray}{\lstinline{<input type = email>}}. This might result in a false positive in certain forms, like the one shown in Listing~\ref{code:false_positive}.

%%        	\begin{lstlisting}[language=HTML,caption={E-Mail field checker
%%               - false positives, the system incorrectly classifies
%%               this as an e-mail form.},label={code:false_positive}, float]
%% <form method="post">
%% E-Mail us if you have any questions!!
%% <input type="text" name="query"><br>
%% <input type="submit" value="Search">
%% </form>
%%        	\end{lstlisting}

%%        	The word \texttt{E-Mail} on Line 2 will result in our system classifying this form as a potential e-mail form, while it clearly is not. However, as we will see, this is not really a significant issue, as despite being added to the \texttt{email\_forms} table, this form will never be injected in the `fuzzer' due to the absence of the appropriate input field in the form. We chose to go with this design, as it allows us to detect almost every form that provides the capability to send or receive e-mail.

%\begin{lstlisting}[language=HTML,caption={e-mail field in a different language - French.},label={code:htmlfrench}, literate=%
%	{é}{{\'e}}1, float]
%<input type="text" 
%placeholder="courrier électronique"
%name="courrier_électronique">
%\end{lstlisting}

\section{Assumptions}
We made certain assumptions while building the system. This section describes the assumptions and explores to what extent these hold true:
\begin{enumerate}
	\item \textbf{Crawler is not blocked by firewalls}\\
	This is a requisite for our system to work. If the Crawler is blocked for any reason, we do not get the data feed for our system, and without this input, it is almost impossible to set our system up.
	\item \textbf{The Crawler feed is an ideal representation of the World Wide Web} \\
	This is a reasonable expectation, albeit an unrealistic one.
	
	It is unrealistic because Crawlers work on the concept of proximity. They detect for the presence of In-Links and Out-Links from a particular URL, and hence the returned URLs are usually related to each other (at least the ones that are returned adjacent to each other).
	
	However, this assumption is reasonable due to the `Law of averages' \cite{wiki:Law_of_averages}, the `Law of big numbers' \cite{wiki:Law_of_large_numbers}, and the concept of `Regression to the mean' \cite{wiki:Regression_toward_the_mean}. Simply stated, a crawl of this large magnitude should give us a very distributed sample of the overall Web, eventually converging to the average of all websites in existence.
	
	\item \textbf{Injection of \dq{\texttt{bcc}} indicates the existence of E-Mail Header Injection Vulnerability} \\
	We assume that the ability to inject a \dq{\texttt{bcc}} header field is proof that the E-Mail Header Injection vulnerability exists in the application. We do not inject any additional payloads that can modify the subject, message body, etc.\ as this analysis is designed to be as benign as possible.
	We believe that this is a reasonable assumption, as altering e-mail headers is a goal of exploiting E-Mail Header Injection vulnerability.
\end{enumerate}

That concludes our discussion about the design of the system. To recap, we discussed our approach, the system architecture and how the components fit into our architecture. We also discussed the issues faced, and the assumptions that we made while building the system. %The next chapter describes, in brief, the experimental setup we used for our system.
\subsection{Mitigation Strategy}
\label{disc:mitigation}
After demonstrating that \ehi vulnerabilities exist on the web at large, we now describe the most common measures that can be taken to prevent the occurrence of this vulnerability, or at least reduce the impact.

Using a safe, well tested \email library is the preferred way of
preventing \ehi vulnerabilities (removing the burden of input
sanitization from the developer). A list of known secure libraries for
each language and framework discussed is shown in
Table~\ref{tab:maillib}.
	\begin{table}[tbp]
		\centering
		\scriptsize
		\begin{tabular}{|l|l|}
			\hline
			\multicolumn{1}{|c|}{\textbf{Language}} &
			\multicolumn{1}{c|}{\textbf{Mail Libraries}} \\
			\hline
			PHP & {{PEAR Mail\cite{Hagenbuch2016}, PHPMailer\cite{PHPMailer2016}, Swiftmailer\cite{SwiftMailer2016}}}\\
			\hline
			Python & SMTPLib with email.header.Header\\
			\hline
			Java & Apache Commons E-Mail\cite{ACE2016}\\
			\hline
			Ruby & Ruby Mail \textgreater{}= 2.6\cite{RubyMailGem2016}\\
			\hline
			WordPress & Contact Form 7\cite{CF7}\\
			\hline
		\end{tabular}
		\caption[\titlecap{Mail libraries that prevent e-mail header
            injection}]{Mail libraries that prevent e-mail header
          injection.}
        \vspace{-5ex}        
		\label{tab:maillib}
	\end{table}

Content management systems such as WordPress and Drupal include
libraries and plugins to prevent \ehi. Thus, websites built with such
CMS' are usually resistant to these attacks. However, it is advised to
use the correct \email plugin, as not all plugins might be secure. An example of a secure plugin is shown in Table~\ref{tab:maillib}.

If neither of the two options are feasible (in-house
production, or lack of support infrastructure), developers can
choose to perform proper input sanitization. Sanitization should be
done with RFC5322~\cite{rfc5322} in mind to
ensure that all edge cases are covered.
	
	%% Client Side validation alone is not sufficient, and must be supplemented by server-side validation to mitigate the attack. Constant updates to validation methods are required so that new attack vectors do not harm the website in any way.
	%% Test driven development for such validation methods is also encouraged so that we can be reasonably sure of our defense mechanisms.



	
	\section{Related Work}

% Adam: Sai, can you combine all the \cites at the same line like I did for the others? 
There are different approaches to finding vulnerabilities in web applications, and most approaches use either Black-Box testing or White-Box testing.
Our work is based on the black-box testing approach to finding vulnerabilities on websites, and research has made use of this methodology to find vulnerabilities in web applications~\cite{Beizer:1995:BTT:202699, Huang, kals2006secubat, payet13:ears-in-the-wild, zanero2005automatic}. There has been significant discussion on both the benefits of such an approach~\cite{black-box} and its shortcomings~\cite{Doupe2012, Doupe2010}.

Our work does not intend to act as a vulnerability scanner, but as a means to identify an \ehi vulnerability in a given web application. In this sense, because we are injecting payloads into the web application, our work is related to other injection based attacks, such as SQL Injection~\cite{sql1, sql0, sql2}, Cross-Site Scripting \cite{Injection1, KleinAmit}, HTTP Header Injection~\cite{sessionride}, and the related Simple Mail Transfer Protocol (SMTP) Injection~\cite{Terada2015}.

The attack described by Terada~\cite{Terada2015} is one that attacks the underlying SMTP mail servers by injecting SMTP commands (which are closely related to E-Mail Headers and usually have a one-to-one mapping, e.g., \texttt{To} e-mail header has a corresponding \texttt{To} SMTP header) to exploit the SMTP server's pipelining mechanism. Terada also describes proof-of-concept attacks against certain mailing libraries such as \texttt{Ruby Mail} and \texttt{JavaMail}. This attack, although trying to achieve a similar result, is distinctly different from ours. Terada's paper also makes this observation and discusses why it is different from \ehi.

In comparison, our work tries to exploit application-level flaws in user input sanitization, which allow this attack. Our work does not intend to exploit the pipelining mechanism, but to exploit the implementation of the mail function in most popular programming languages, which leaves them with no way to distinguish between user supplied headers and headers that are legitimately added by the application.

Although \ehi vulnerabilities have been present for over a decade, there has not been much written about it in the literature, and we find only a few articles on the Internet describing the attack.

The first documented article dates to over a decade ago; a late 2004 article on phpsecure.info~\cite{Tobozo} accredited to user \lstinline|tobozo| describing how this vulnerability existed in the reference implementation of the mail function in PHP, and how it can be exploited. Following this, we found other blog posts~\cite{Calin, DK, Injection2, Nicol, Pope}, each describing how to exploit the vulnerability by using newlines to camouflage headers inside user input. A wiki entry~\cite{Injection} also describes the ways to prevent such an attack. However, none of these articles have performed these attacks against real-life websites.

Another blog post written by user \lstinline|Voxel@Night|~\cite{Tendencies2014}, recounts an actual attack against a WordPress plugin, \texttt{Contact Form}, with a proof of concept\footnotemark. It also showcases the vulnerable code in the plugin that causes the vulnerability. However, this article targets just one plugin and does not aim to find the prevalence of said plugin usage. Neither does it inform the creators of the plugin to fix the discovered vulnerability. Note that this plugin is used actively on 300,000 websites (according to~\cite{BestWebSoft2016}), but is yet to be fixed.
\ehi vulnerability was described briefly by Stuttard and Pinto in their book, ``\emph{The Web Application Hacker's Handbook}''~\cite{stuttard2011web}. The book, however, does not go into detail on either the attack or the ways to mitigate such an attack. Our work, on the other hand discusses the means to mitigate the attack. We also describe, in detail, the payloads that can be used and the need for varying the payloads (Section~\ref{Comp:Fuzzer}).

To the best of our knowledge, no other research has been conducted to determine the prevalence of this vulnerability on the World Wide Web. We have managed to, on a large scale, crawl and inject web applications with comparatively benign payloads (the \texttt{bcc} header) to identify the existence of this vulnerability without causing any ostensible harm to the website. Our injected payloads \emph{do not contain any special characters other than the newline character} and thus can't cause any unintended consequences. Also, as we are only injecting the \texttt{bcc} header, the underlying mail servers should not be affected by the additional load. Our work serves to not only prove the existence of the vulnerability on the World Wide Web but to quantify its prevalence.

	
	\section{Conclusions}
We have showcased a novel approach involving black-box testing to identify the presence of \ehi in a web application. Using this approach, we have demonstrated that our system was able to crawl \urls\ web pages finding \forms\ forms, out of which \emailforms\ forms were fuzzable. We fuzzed \fuzzed\ forms and found \recd\  forms that allowed us to send/receive e-mails. Out of these, we were able to inject malicious payloads into \malfuzzed\ forms, identifying \success\ vulnerable forms (\successDelta\ success rate). This indicates that the vulnerability is widespread, and needs attention from both web application and library developers. 

We hope that our work sheds light on the prevalence of this vulnerability and that it ensures that the implementation of the \texttt{mail} function in popular languages is fixed to differentiate between User-supplied headers, and headers that are legitimately added by the application.

	
	%
	% The following two commands are all you need in the
	% initial runs of your .tex file to
	% produce the bibliography for the citations in your paper.
	\bibliographystyle{abbrv}
	{\scriptsize\bibliography{biblio}}
%	\bibliography{biblio}  
	% You must have a proper ".bib" file
	%  and remember to run:
	% latex bibtex latex latex
	% to resolve all references
	%
	%\balancecolumns
\end{document}
