\subsection{Responsible Disclosure of Discovered Vulnerabilities}
After we discovered an \ehi vulnerability on a particular website, we attempted to notify the developers of the vulnerable web application, along with a brief description of the vulnerability.
We chose to \email the following mailboxes, following the rules specified in RFC~2142~\cite{rfc2142}:
% Adam: don't we use \texttt for all the email addresses in the paper? We should use the same thing here - DONE.
\begin{itemize}
	\item \texttt{security@domain.com} - Used for Security bulletins or queries.
	\item \texttt{admin@domain.com} - Used to contact the administrator of a website.
	\item \texttt{webmaster@domain.com} - Synonym for administrator, same functionality as admin.
\end{itemize}

% Adam: sai, is this number up-to-date? 
Out of the \domains\ vulnerable domains found, only \emailedDefaultmailbox websites had the mailboxes able to receive \emails. For the remaining domains, we used the \texttt{whois}~\cite{whois} data to find the contact details of the owner and then \emailed them. The number of emails we sent and the number of developer responses we received is shown in Table~\ref{tab:devresp}.

\begin{table}[tbp]
\centering
\normalsize
\begin{tabular}{|c|c|c|}
	\hline
	\multicolumn{1}{|p{2cm}}{\centering \textbf{Notified websites}} &
	\multicolumn{1}{|p{2cm}|}{\centering \textbf{Developer Responses}} &
	\multicolumn{1}{p{2cm}|}{\centering \textbf{Confirmed discoveries}}\\
	\hline
	\domains\ & \responses & \confirmed \\
	\hline
\end{tabular}
	\caption[\titlecap{}]{Responsible disclosure of the discovered vulnerabilities to developers and the number of received responses.}
	\label{tab:devresp}
\end{table}

% Adam: did we get updated notifications? - nope. DONE.
We received \responses developer responses, confirming \confirmed discovered vulnerabilities. Three of the developers fixed the vulnerability on their website.
% is this done? 1 additional email asking for more information, but no confirmation.
From our research, it is clear that \ehi is quite widespread as a vulnerability, appearing on \successDelta\ of forms that we were able to perform automated attacks on. This value acts as a \emph{lower bound} for prevalence of \ehi vulnerability, and can quite easily be larger if the attacks were broader, crafted for the individual web application, and  less automated. 
