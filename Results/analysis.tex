\section[Analysis of Data]{Our Analysis of the received e-mail Data}
During our analysis of the received e-mails, we found that the e-mails that we received belonged to one of three broad categories:
\begin{enumerate}
	\item E-Mails with the \dq{\texttt{bcc}} header successfully injected\\
	This form of injection was our initial objective and we found 265 such e-mails in our received e-mails. This indicates that the websites that sent out these e-mails are vulnerable to e-mail header injection, where we could inject and manipulate any header.
	
	\item E-Mails with the \dq{\texttt{to}} header successfully injected\\
	We discovered an unintended vulnerability which we would like to christen `TO~header injection'. These injections reflect the ability to inject any number of e-mail addresses into the \dq{\texttt{to}} field while being unable to inject any other header into the e-mails. We attribute this behavior to inconsistent sanitization by the application. 
	The vulnerability is further aided by the leniency shown by mail servers, wherein they parsed malformed e-mail addresses and delivered it to the right mail server, and on the receiving end, the mail was delivered to the right mailbox. 
	
	While not allowing us complete control over the e-mails sent, `TO header injection' makes it possible to append any number of e-mail addresses, thereby enabling us to leak information, and/or perform DoS (Denial of Service) attacks.
	
	\item E-Mails with the \dq{\texttt{x-check}} header successfully injected\\
    The third category of e-mails received were e-mails with the \dq{\texttt{x-check}} header injected. As discussed in Section~\ref{analyze:detect_x_check}, 
    these let us differentiate between unsuccessful attempts and successful attempts by injecting the additional header, allowing us to check whether headers other than the \dq{\texttt{bcc}} header can be injected into the generated e-mail. 
\end{enumerate}
We list each category and the number of e-mails received by the category in Table~\ref{tab:analysis}. 

\begin{table}[tbp]
	\centering
	\scriptsize
	\begin{tabular}{|p{.32\textwidth}|p{.13\textwidth}|}
		\hline
		\textbf{Type of Injection} & \textbf{No. of e-mails received}\\
		\hline
		E-Mail Header Injections with \texttt{bcc} header & \ehibcc \\
		\hline
		E-Mail Header Injections with \texttt{x-check} header & \ehixcheck \\
		\hline
		\texttt{To header} injections alone & \ehito \\
		\hline
		Injections with both \texttt{bcc} and \texttt{x-check} headers & \ehibccxcheck \\
		\hline
		Both \texttt{To header} injections and x-check headers &
		\ehitoxcheck \\
		\hline
		\texttt{x-check} headers found in \texttt{nuser} e-mails & \ehinuserxcheck \\
		\hline
		Unique \texttt{x-check} headers found in \texttt{nuser} e-mails & \ehiuniquenuserxcheck \\
		\hline
		Total successful injections (1 + 3 + 7) & \success \\
		
		\hline
	\end{tabular}
	\caption[\titlecap{Analysis of the data}]{Classification of the e-mails that we received into broad categories of the vulnerability.}
	\label{tab:analysis}
\end{table}

We explain the combination of these header injections (4-7) as follows:
\begin{itemize}
	\item E-Mail Header Injections with \dq{\texttt{bcc}} and \dq{\texttt{x-check}} headers\\
	These represent the perfect attack scenario where we are able to inject multiple headers into the e-mails. We can see that over 75\% of the received \dq{\texttt{bcc}} header injected e-mails are also susceptible to other header injections.
	
	\item Both \dq{\texttt{To}} header injections and \dq{\texttt{x-check}} headers \\
	This combination shows us that in addition to being able to inject into the \dq{\texttt{To}} fields, we are able to inject additional headers into the e-mail. It is not clear what causes this behavior; however, these can be exploited to achieve the same result as a regular E-Mail Header Injection.
	
	\item \dq{\texttt{x-check}} headers found in `nuser' e-mails\\
	In addition to analyzing the `maluser' account, we also analyze emails received by the `nuser' account. We explain the presence of these headers in the following paragraph.

	\item Unique \dq{\texttt{x-check}} headers found in `nuser' e-mails\\
	These represent the e-mails with \lstinline|form_ids| that were \emph{not} already found in the `maluser' account. We attribute these e-mails to (probably) having a backend that was built with Python or another language having a similar behavior with respect to constructing headers.
	
	\item Total successful injections\\
	This represents the total number of successful injections our system made. This includes the E-Mail Header Injections with \dq{\texttt{bcc}} header\,(1), \dq{\texttt{To}} header injections alone\,(3), and Unique \dq{\texttt{x-check}} headers found in `nuser' e-mails\,(7). This is the total number of vulnerabilities that were found by our system.
	
\end{itemize}
